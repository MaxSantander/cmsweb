%% Generated by Sphinx.
\def\sphinxdocclass{report}
\documentclass[a4paper,12pt,ngerman]{sphinxmanual}
\ifdefined\pdfpxdimen
   \let\sphinxpxdimen\pdfpxdimen\else\newdimen\sphinxpxdimen
\fi \sphinxpxdimen=.75bp\relax
\ifdefined\pdfimageresolution
    \pdfimageresolution= \numexpr \dimexpr1in\relax/\sphinxpxdimen\relax
\fi
%% let collapsible pdf bookmarks panel have high depth per default
\PassOptionsToPackage{bookmarksdepth=5}{hyperref}

\PassOptionsToPackage{booktabs}{sphinx}
\PassOptionsToPackage{colorrows}{sphinx}

\PassOptionsToPackage{warn}{textcomp}
\usepackage[utf8]{inputenc}
\ifdefined\DeclareUnicodeCharacter
% support both utf8 and utf8x syntaxes
  \ifdefined\DeclareUnicodeCharacterAsOptional
    \def\sphinxDUC#1{\DeclareUnicodeCharacter{"#1}}
  \else
    \let\sphinxDUC\DeclareUnicodeCharacter
  \fi
  \sphinxDUC{00A0}{\nobreakspace}
  \sphinxDUC{2500}{\sphinxunichar{2500}}
  \sphinxDUC{2502}{\sphinxunichar{2502}}
  \sphinxDUC{2514}{\sphinxunichar{2514}}
  \sphinxDUC{251C}{\sphinxunichar{251C}}
  \sphinxDUC{2572}{\textbackslash}
\fi
\usepackage{cmap}
\usepackage[T1]{fontenc}
\usepackage{amsmath,amssymb,amstext}
\usepackage[ngerman]{babel}



\usepackage{tgtermes}
\usepackage{tgheros}
\renewcommand{\ttdefault}{txtt}



\usepackage[Sonny]{fncychap}
\ChNameVar{\Large\normalfont\sffamily}
\ChTitleVar{\Large\normalfont\sffamily}
\usepackage{sphinx}

\fvset{fontsize=auto}
\usepackage{geometry}


% Include hyperref last.
\usepackage{hyperref}
% Fix anchor placement for figures with captions.
\usepackage{hypcap}% it must be loaded after hyperref.
% Set up styles of URL: it should be placed after hyperref.
\urlstyle{same}

\addto\captionsngerman{\renewcommand{\contentsname}{Hauptinhalt}}

\usepackage{sphinxmessages}
\setcounter{tocdepth}{1}


        \usepackage[T1]{fontenc}
        \usepackage[utf8]{inputenc}
    

\title{RIS Backend Dokumentation}
\date{30.12.2024}
\release{1.0}
\author{Maximiliano Santander}
\newcommand{\sphinxlogo}{\vbox{}}
\renewcommand{\releasename}{Release}
\makeindex
\begin{document}

\ifdefined\shorthandoff
  \ifnum\catcode`\=\string=\active\shorthandoff{=}\fi
  \ifnum\catcode`\"=\active\shorthandoff{"}\fi
\fi

\pagestyle{empty}
\sphinxmaketitle
\pagestyle{plain}
\sphinxtableofcontents
\pagestyle{normal}
\phantomsection\label{\detokenize{index::doc}}


\sphinxAtStartPar
Willkommen zur Dokumentation des RIS Backend\sphinxhyphen{}Projekts. Diese Dokumentation bietet einen Überblick über die Planung, den Entwurf, die Implementierung und die Tests des Projekts.


\chapter{Inhaltsverzeichnis}
\label{\detokenize{index:inhaltsverzeichnis}}
\sphinxstepscope


\section{Projektplanung und Analyse}
\label{\detokenize{sections/projektplanung_analyse:projektplanung-und-analyse}}\label{\detokenize{sections/projektplanung_analyse::doc}}

\subsection{Einleitung}
\label{\detokenize{sections/projektplanung_analyse:einleitung}}

\subsubsection{Projektumfeld}
\label{\detokenize{sections/projektplanung_analyse:projektumfeld}}\begin{itemize}
\item {} 
\sphinxAtStartPar
Ich mache zurzeit eine Umschulung zum Fachinformatiker für Anwendungsentwicklung beim Städtische Berufsschule III Matthäus Runtinger in Regensburg.

\item {} 
\sphinxAtStartPar
Aktuell absolviere ich meine Ausbildung als Fachinformatiker für Anwendungsentwicklung bei der Firma RIS Web\sphinxhyphen{} \& Software\sphinxhyphen{}Development GmbH \& Co. KG in Regensburg, nachfolgend RIS genannt.

\item {} 
\sphinxAtStartPar
Der Betrieb beschäftigt 13 Mitarbeiter. RIS Development ist ein offizieller Servicepartner der E\sphinxhyphen{}Commerce\sphinxhyphen{}Software von JTL.

\item {} 
\sphinxAtStartPar
Neben dem Support für die JTL\sphinxhyphen{}Software werden auch individuelle Komponenten für den Webshop oder die Warenwirtschaft (WAWI) entwickelt. Diese reichen vom Erstellen von Template\sphinxhyphen{}Vorlagen für Dokumente und Webdesign bis hin zu erweiterten Funktionalitäten in Form von Shop\sphinxhyphen{}Plugins.

\item {} 
\sphinxAtStartPar
Es werden auch individuelle Lösungen für Kunden außerhalb der E\sphinxhyphen{}Commerce\sphinxhyphen{}Software erstellt. Der Umfang reicht vom Entwickeln von Corporate Designs, Social\sphinxhyphen{}Media\sphinxhyphen{}Marketing, Neugestaltung des Website\sphinxhyphen{}Designs, Erstellung von Web\sphinxhyphen{}Applikationen bis hin zur Umsetzung nativer Software.

\end{itemize}


\subsubsection{Projektziel}
\label{\detokenize{sections/projektplanung_analyse:projektziel}}
\sphinxAtStartPar
Das Ziel des Projekts ist der Aufbau und die Implementierung eines Backends für RIS. Moderne Technologien wie Django, Laravel oder Node.js werden in Betracht gezogen. Dieses Backend soll folgende Anforderungen erfüllen:
\begin{enumerate}
\sphinxsetlistlabels{\arabic}{enumi}{enumii}{}{.}%
\item {} 
\sphinxAtStartPar
\sphinxstylestrong{Volle Kontrolle}:
\sphinxhyphen{} Alle Backend\sphinxhyphen{}Funktionen sollen feinjustiert werden und unabhängig von allgemeinen Lösungen wie WordPress arbeiten.
\sphinxhyphen{} Die Nutzung von Templates und Plugins, die oft mit Abogebühren verbunden sind, soll vermieden werden.

\item {} 
\sphinxAtStartPar
\sphinxstylestrong{Verschlankung}:
\sphinxhyphen{} Das Backend soll gezielt auf die spezifischen Bedürfnisse von RIS zugeschnitten werden, um unnötige Funktionen zu vermeiden und die Bedienbarkeit zu vereinfachen.

\item {} 
\sphinxAtStartPar
\sphinxstylestrong{Effiziente Inhaltspflege}:
\sphinxhyphen{} Die Verwaltung von Inhalten soll ohne die Einschränkungen und Komplexität eines WordPress\sphinxhyphen{}Templates erfolgen, wodurch ein direkter und effizienter Arbeitsprozess ermöglicht wird.

\item {} 
\sphinxAtStartPar
\sphinxstylestrong{Performance}:
\sphinxhyphen{} Die Website soll serverseitig gerendert werden, und die Seiteninhalte sollen vollständig im Cache gespeichert werden, um mit neuer Server\sphinxhyphen{}Hardware optimale Ladegeschwindigkeiten zu erreichen.

\end{enumerate}


\subsubsection{Ausgangssituation}
\label{\detokenize{sections/projektplanung_analyse:ausgangssituation}}
\sphinxAtStartPar
Die aktuelle Website von RIS Web\sphinxhyphen{} \& Software\sphinxhyphen{}Development GmbH \& Co. KG, im Folgenden RIS genannt, basiert auf WordPress und erfüllt ihre grundlegende Funktion. Jedoch sind wir mit WordPress auf allgemeine Lösungen angewiesen, die nicht immer unseren spezifischen Anforderungen entsprechen:
\begin{itemize}
\item {} 
\sphinxAtStartPar
WordPress erfordert die Nutzung allgemeiner Plugins, die oft nicht an die Anforderungen von RIS angepasst sind.

\item {} 
\sphinxAtStartPar
WordPress enthält viele Funktionen, die nicht benötigt werden, was unnötigen Overhead beim Lernen und Bedienen verursacht.

\item {} 
\sphinxAtStartPar
Die Verwaltung von Inhalten in WordPress mit Templating hat sich als umständlich und unvorhersehbar erwiesen.

\end{itemize}


\subsection{Abgrenzung des Aufgabenbereichs}
\label{\detokenize{sections/projektplanung_analyse:abgrenzung-des-aufgabenbereichs}}
\sphinxAtStartPar
Der Fokus des Projekts liegt auf der Entwicklung des Backends. Frontend\sphinxhyphen{}Aufgaben sowie Optimierungen in Bereichen wie Blogs sind nicht Teil des Projekts, da ein Kollege sich um das Frontend kümmern wird. Außerdem werden die Sicherheitsmaßnahmen als Standard angesehen, da nur minimale Daten von den Kunden abgefragt werden (z.B. keine Zahlungsinformationen).


\subsection{Projektplanung}
\label{\detokenize{sections/projektplanung_analyse:projektplanung}}

\subsubsection{Projektphasen}
\label{\detokenize{sections/projektplanung_analyse:projektphasen}}\begin{itemize}
\item {} 
\sphinxAtStartPar
Das Projekt beginnt am 11. November 2024 und endet am 6. Januar 2025, was insgesamt etwa 8 Wochen umfasst. Innerhalb dieses Zeitraums werden die verschiedenen Phasen des Projekts parallel zu den regulären Aufgaben des Unternehmens entwickelt.

\item {} 
\sphinxAtStartPar
Die projektbezogenen Tätigkeiten werden so geplant, dass die 80 Stunden für das Projekt effizient genutzt werden, um die Ziele fristgerecht zu erreichen.

\end{itemize}


\subsubsection{Grobe Zeitplanung}
\label{\detokenize{sections/projektplanung_analyse:grobe-zeitplanung}}

\begin{savenotes}\sphinxattablestart
\sphinxthistablewithglobalstyle
\centering
\begin{tabulary}{\linewidth}[t]{TT}
\sphinxtoprule
\sphinxstyletheadfamily 
\sphinxAtStartPar
Projektphase
&\sphinxstyletheadfamily 
\sphinxAtStartPar
Geplante Zeit
\\
\sphinxmidrule
\sphinxtableatstartofbodyhook
\sphinxAtStartPar
Projektplanung/Analyse
&
\sphinxAtStartPar
8 h
\\
\sphinxhline
\sphinxAtStartPar
Entwurf
&
\sphinxAtStartPar
16 h
\\
\sphinxhline
\sphinxAtStartPar
Implementierung
&
\sphinxAtStartPar
36 h
\\
\sphinxhline
\sphinxAtStartPar
Test/Durchführung
&
\sphinxAtStartPar
8 h
\\
\sphinxhline
\sphinxAtStartPar
Dokumentation
&
\sphinxAtStartPar
8 h
\\
\sphinxhline
\sphinxAtStartPar
Abnahme
&
\sphinxAtStartPar
4 h
\\
\sphinxhline
\sphinxAtStartPar
\sphinxstylestrong{Gesamt}
&
\sphinxAtStartPar
\sphinxstylestrong{80 h}
\\
\sphinxbottomrule
\end{tabulary}
\sphinxtableafterendhook\par
\sphinxattableend\end{savenotes}

\sphinxAtStartPar
{\hyperref[\detokenize{sections/tables:detalierter-zeitplanung}]{\sphinxcrossref{\DUrole{std}{\DUrole{std-ref}{Detallierter Zeitplanung}}}}}


\subsubsection{Ressourcenplanung}
\label{\detokenize{sections/projektplanung_analyse:ressourcenplanung}}
\sphinxAtStartPar
\sphinxstylestrong{Software:}
\begin{itemize}
\item {} 
\sphinxAtStartPar
Framework: Django

\item {} 
\sphinxAtStartPar
Django Tools: z.B. Pillow

\item {} 
\sphinxAtStartPar
DB\sphinxhyphen{}Software: SQLite für Entwicklung, PostgreSQL für Produktion

\item {} 
\sphinxAtStartPar
IDE: PyCharm

\item {} 
\sphinxAtStartPar
Cache\sphinxhyphen{}Tools: Redis, Middleware

\item {} 
\sphinxAtStartPar
Test\sphinxhyphen{}Software: PyTest Django, PyTest, FactoryBoy (Testdaten Erstellung), pdoc

\item {} 
\sphinxAtStartPar
Versionverwaltung: GitLab

\end{itemize}

\sphinxAtStartPar
\sphinxstylestrong{Hardware:}
\sphinxhyphen{} Laptop


\subsection{Analysephase}
\label{\detokenize{sections/projektplanung_analyse:analysephase}}

\subsubsection{Auswahl der Technologieplattform}
\label{\detokenize{sections/projektplanung_analyse:auswahl-der-technologieplattform}}
\sphinxAtStartPar
Für das Projekt wurden moderne Technologien wie Django, Laravel und Node.js in Betracht gezogen, wobei wir uns nach gründlicher Recherche für Django entschieden haben. Dieses Framework bietet robuste und skalierbare Backend\sphinxhyphen{}Lösungen und passt ideal zu den Anforderungen unseres Projekts. Ich und der Rest unseres Entwicklungsteams haben bereits umfassende Kenntnisse in Python sowie in JavaScript, HTML und CSS.

\sphinxAtStartPar
\sphinxstylestrong{Vorteile von Django}
\begin{itemize}
\item {} 
\sphinxAtStartPar
\sphinxstylestrong{Caching}: Ein flexibles Caching\sphinxhyphen{}System unterstützt Technologien wie Middleware Cache und Redis (Backend Cache), was die Anwendungsperformance erheblich steigert.

\item {} 
\sphinxAtStartPar
\sphinxstylestrong{Vollständiges Framework}: Django ist ein „batteries\sphinxhyphen{}included“ Framework mit integrierten Funktionen wie Authentifizierung, Administrationsoberflächen und einem leistungsstarken ORM, was die Entwicklung effizienter gestaltet.

\item {} 
\sphinxAtStartPar
\sphinxstylestrong{Große Community und Dokumentation}: Die aktive Community und die umfassende Dokumentation erleichtern die Problemlösung und kontinuierliche Wartung.

\end{itemize}

\sphinxAtStartPar
In der Entwicklungsphase nutzen wir SQLite wegen seiner Einfachheit und weil es schon by Default in Django konfiguriert ist. In der Produktionsphase setzen wir PostgreSQL ein, um von den erweiterten Funktionen, der Leistung und der Skalierbarkeit zu profitieren, die unser Projekt benötigt.


\subsubsection{Ist\sphinxhyphen{}Analyse}
\label{\detokenize{sections/projektplanung_analyse:ist-analyse}}\begin{itemize}
\item {} 
\sphinxAtStartPar
Das Projekt findet im Rahmen der Webentwicklung bei der RIS statt. Ziel ist es, ein neues, maßgeschneidertes Backend für die Unternehmenswebsite zu entwickeln, um die Abhängigkeit von WordPress zu vermeiden.

\item {} 
\sphinxAtStartPar
Die bestehende Website basiert auf WordPress, was zu Problemen wie unnötigem Overhead, langsamerer Performance und umständlicher Inhaltsverwaltung führt. Die derzeit verwendeten Plugins und Vorlagen entsprechen nicht den spezifischen Anforderungen von RIS.

\end{itemize}


\subsubsection{Wirtschaftlichkeitsanalyse}
\label{\detokenize{sections/projektplanung_analyse:wirtschaftlichkeitsanalyse}}
\sphinxAtStartPar
Eine klassische Wirtschaftlichkeitsanalyse, inklusive Projektkosten und Amortisationsdauer,
wurde für dieses Projekt aus folgenden Gründen nicht durchgeführt:

\sphinxAtStartPar
\sphinxstylestrong{Ausbildungskontext}
\sphinxhyphen{} Projekt im Rahmen der Berufsausbildung
\sphinxhyphen{} Fokus auf Kompetenzerwerb statt wirtschaftlichem Nutzen
\sphinxhyphen{} Keine zusätzlichen Personal\sphinxhyphen{} oder Ausbildungskosten

\sphinxAtStartPar
\sphinxstylestrong{Ressourcen und Kosten}
\sphinxhyphen{} Nutzung vorhandener Infrastruktur (Hardware/Software)
\sphinxhyphen{} Einsatz von Open\sphinxhyphen{}Source\sphinxhyphen{}Technologien (Django, Python)
\sphinxhyphen{} Keine externen Ressourcen oder Lizenzen erforderlich
\sphinxhyphen{} Integration in bestehende IT\sphinxhyphen{}Systeme

\sphinxAtStartPar
\sphinxstylestrong{Projektrahmen}
\sphinxhyphen{} Entwicklung durch Auszubildenden
\sphinxhyphen{} Teil der regulären Ausbildungszeit
\sphinxhyphen{} Keine zusätzlichen Betriebskosten

\sphinxAtStartPar
Die Projektbewertung konzentrierte sich stattdessen auf:
\begin{itemize}
\item {} 
\sphinxAtStartPar
Technische Umsetzung

\item {} 
\sphinxAtStartPar
Code\sphinxhyphen{}Qualität

\item {} 
\sphinxAtStartPar
Dokumentation

\item {} 
\sphinxAtStartPar
Lernerfahrung

\end{itemize}


\subsection{Vorgehensmodelle}
\label{\detokenize{sections/projektplanung_analyse:vorgehensmodelle}}
\sphinxAtStartPar
Für dieses Projekt wurde das Wasserfallmodell gewählt, da es optimal zu den folgenden Projektbedingungen passt:
\begin{itemize}
\item {} 
\sphinxAtStartPar
\sphinxstylestrong{Klare Zeitplanung}:
\begin{itemize}
\item {} 
\sphinxAtStartPar
Fester Zeitrahmen von 80 Stunden

\item {} 
\sphinxAtStartPar
Klar definierte Projektphasen mit zugewiesenen Zeitbudgets

\end{itemize}

\item {} 
\sphinxAtStartPar
\sphinxstylestrong{Einzelentwickler\sphinxhyphen{}Projekt}:
\begin{itemize}
\item {} 
\sphinxAtStartPar
Keine Notwendigkeit für komplexe Team\sphinxhyphen{}Koordination

\item {} 
\sphinxAtStartPar
Direkte Kontrolle über alle Entwicklungsphasen

\item {} 
\sphinxAtStartPar
Vereinfachte Entscheidungsprozesse

\end{itemize}

\item {} 
\sphinxAtStartPar
\sphinxstylestrong{Vordefinierte Ressourcen}:
\begin{itemize}
\item {} 
\sphinxAtStartPar
Festgelegte technische Infrastruktur

\item {} 
\sphinxAtStartPar
Bereits bestimmte Entwicklungswerkzeuge

\item {} 
\sphinxAtStartPar
Klare Hardware\sphinxhyphen{}Anforderungen

\end{itemize}

\item {} 
\sphinxAtStartPar
\sphinxstylestrong{Spezifisches Projektziel}:
\begin{itemize}
\item {} 
\sphinxAtStartPar
Eindeutig definierter Funktionsumfang

\item {} 
\sphinxAtStartPar
Klare Abgrenzung der Anforderungen

\item {} 
\sphinxAtStartPar
Vorhersehbare technische Umsetzung

\end{itemize}

\end{itemize}

\sphinxAtStartPar
Diese Rahmenbedingungen ermöglichen eine sequentielle Abarbeitung der Projektphasen, wie sie im Wasserfallmodell vorgesehen ist.


\subsection{Make or Buy\sphinxhyphen{}Entscheidung}
\label{\detokenize{sections/projektplanung_analyse:make-or-buy-entscheidung}}\begin{itemize}
\item {} 
\sphinxAtStartPar
Als weborientiertes Entwicklungsunternehmen war die Entscheidung klar, eine fertige Website\sphinxhyphen{}Lösung zu kaufen, kam nicht in Frage. Stattdessen haben wir uns dafür entschieden, ein eigenes Backend zu entwickeln, um volle Kontrolle über alle Funktionen und Anpassungen zu haben. Diese Entscheidung geht jedoch über eine einfache Neugestaltung hinaus. Es ging darum, eine Grundlage für eine effiziente, skalierbare und langfristig wartbare Lösung zu schaffen, die den spezifischen Anforderungen von RIS gerecht wird.

\end{itemize}


\subsection{Nutzwertanalyse}
\label{\detokenize{sections/projektplanung_analyse:nutzwertanalyse}}

\begin{savenotes}\sphinxattablestart
\sphinxthistablewithglobalstyle
\centering
\begin{tabulary}{\linewidth}[t]{TTTTT}
\sphinxtoprule
\sphinxstyletheadfamily 
\sphinxAtStartPar
Kriterium
&\sphinxstyletheadfamily 
\sphinxAtStartPar
Gewichtung
&\sphinxstyletheadfamily 
\sphinxAtStartPar
Django
&\sphinxstyletheadfamily 
\sphinxAtStartPar
Laravel
&\sphinxstyletheadfamily 
\sphinxAtStartPar
Node.js
\\
\sphinxmidrule
\sphinxtableatstartofbodyhook
\sphinxAtStartPar
Template\sphinxhyphen{}System \& Caching
&
\sphinxAtStartPar
30\%
&
\sphinxAtStartPar
10 (3.0)
&
\sphinxAtStartPar
7 (2.1)
&
\sphinxAtStartPar
6 (1.8)
\\
\sphinxhline
\sphinxAtStartPar
Entwicklungsgeschwindigkeit
&
\sphinxAtStartPar
25\%
&
\sphinxAtStartPar
9 (2.25)
&
\sphinxAtStartPar
8 (2.0)
&
\sphinxAtStartPar
7 (1.75)
\\
\sphinxhline
\sphinxAtStartPar
Database Integration \& ORM
&
\sphinxAtStartPar
25\%
&
\sphinxAtStartPar
9 (2.25)
&
\sphinxAtStartPar
7 (1.75)
&
\sphinxAtStartPar
6 (1.5)
\\
\sphinxhline
\sphinxAtStartPar
Skalierbarkeit
&
\sphinxAtStartPar
20\%
&
\sphinxAtStartPar
8 (1.6)
&
\sphinxAtStartPar
7 (1.4)
&
\sphinxAtStartPar
9 (1.8)
\\
\sphinxhline
\sphinxAtStartPar
\sphinxstylestrong{Gesamtwertung}
&
\sphinxAtStartPar
\sphinxstylestrong{100\%}
&
\sphinxAtStartPar
\sphinxstylestrong{9.1}
&
\sphinxAtStartPar
7.25
&
\sphinxAtStartPar
6.85
\\
\sphinxbottomrule
\end{tabulary}
\sphinxtableafterendhook\par
\sphinxattableend\end{savenotes}

\sphinxAtStartPar
\sphinxstylestrong{Technische Eignung}:
\begin{itemize}
\item {} 
\sphinxAtStartPar
Besonders stark im Bereich Template\sphinxhyphen{}System \& Caching

\item {} 
\sphinxAtStartPar
Ideal für die geforderte serverseitige Renderung

\end{itemize}

\sphinxAtStartPar
\sphinxstylestrong{Teamkompetenz}:
\begin{itemize}
\item {} 
\sphinxAtStartPar
Vorhandene Python/JavaScript\sphinxhyphen{}Expertise außer CSS und HTML im Entwicklungsteam

\end{itemize}

\sphinxAtStartPar
\sphinxstylestrong{Database Integration \& ORM}:
\begin{itemize}
\item {} 
\sphinxAtStartPar
Robustes ORM für komplexe Datenbankstrukturen

\item {} 
\sphinxAtStartPar
Automatische Migrationen und optimierte Queries

\end{itemize}


\subsection{Anwendungsfälle}
\label{\detokenize{sections/projektplanung_analyse:anwendungsfalle}}
\sphinxAtStartPar
\sphinxstylestrong{Administratoren können:}
\begin{itemize}
\item {} 
\sphinxAtStartPar
Inhalt erstellen, bearbeiten und löschen

\end{itemize}

\sphinxAtStartPar
\sphinxstylestrong{Web\sphinxhyphen{}Users können:}
\begin{itemize}
\item {} 
\sphinxAtStartPar
Verschiedene Formulare ausfüllen und schicken

\item {} 
\sphinxAtStartPar
Web surfen

\end{itemize}

\sphinxAtStartPar
Für eine visuelle Darstellung der Anwendungsfälle, siehe {\hyperref[\detokenize{sections/diagramme:use-case-diagram}]{\sphinxcrossref{\DUrole{std}{\DUrole{std-ref}{Use Case Diagramm}}}}}.

\sphinxstepscope


\section{Entwurf}
\label{\detokenize{sections/entwurf:entwurf}}\label{\detokenize{sections/entwurf::doc}}

\subsection{Architekturdesign}
\label{\detokenize{sections/entwurf:architekturdesign}}
\sphinxAtStartPar
Das Backend verwendet die MTV\sphinxhyphen{}Architektur (Model\sphinxhyphen{}Template\sphinxhyphen{}View), die speziell für Django entwickelt wurde. Diese Architektur trennt die Datenlogik (Model), die Präsentationslogik (Template) und die Steuerlogik (View) klar voneinander.


\subsection{MTV\sphinxhyphen{}Architektur}
\label{\detokenize{sections/entwurf:mtv-architektur}}
\sphinxAtStartPar
Siehe {\hyperref[\detokenize{sections/diagramme:mtv-architecture}]{\sphinxcrossref{\DUrole{std}{\DUrole{std-ref}{MTV\sphinxhyphen{}Architektur}}}}} in den Diagrammen für weitere Details.


\subsection{ERD\sphinxhyphen{}Diagramm}
\label{\detokenize{sections/entwurf:erd-diagramm}}
\sphinxAtStartPar
Die Hauptentitäten des ER\sphinxhyphen{}Modells sind:
\begin{itemize}
\item {} 
\sphinxAtStartPar
\sphinxstylestrong{Page}: Verwaltung von mehrsprachigen Inhalten.

\item {} 
\sphinxAtStartPar
\sphinxstylestrong{Block}: Wiederverwendbare Inhaltsblöcke.

\item {} 
\sphinxAtStartPar
\sphinxstylestrong{MenuItem}: Navigationselemente.

\end{itemize}


\subsection{Systemkomponenten}
\label{\detokenize{sections/entwurf:systemkomponenten}}
\sphinxAtStartPar
Die wichtigsten Systemkomponenten umfassen:
\begin{itemize}
\item {} 
\sphinxAtStartPar
\sphinxstylestrong{Page}: Zentrale Entität für Webseiteninhalte, speichert mehrsprachige Inhalte (DE/EN) und verwaltet Meta\sphinxhyphen{}Informationen.

\item {} 
\sphinxAtStartPar
\sphinxstylestrong{Block}: Enthält Template\sphinxhyphen{}basierte Inhaltsblöcke, sortierbare Komponenten und wiederverwendbare Strukturen.

\item {} 
\sphinxAtStartPar
\sphinxstylestrong{MenuItem}: Verwaltet Navigationsstrukturen, template\sphinxhyphen{}basierte Menüelemente und sortierbare Menükomponenten.

\end{itemize}


\subsection{Beziehungen}
\label{\detokenize{sections/entwurf:beziehungen}}\begin{itemize}
\item {} 
\sphinxAtStartPar
\sphinxstylestrong{Page} (n) hat \sphinxstylestrong{Block} (m)

\item {} 
\sphinxAtStartPar
\sphinxstylestrong{Page} (1) hat \sphinxstylestrong{MenuItem} (0..1)

\end{itemize}

\sphinxAtStartPar
Diese Struktur ermöglicht:
\sphinxhyphen{} Flexible Seitenerstellung
\sphinxhyphen{} Wiederverwendbare Komponenten
\sphinxhyphen{} Mehrsprachige Inhalte
\sphinxhyphen{} Geordnete Navigation


\subsection{Datenmodell}
\label{\detokenize{sections/entwurf:datenmodell}}
\sphinxAtStartPar
Das Datenmodell umfasst die folgenden Klassen:

\begin{sphinxVerbatim}[commandchars=\\\{\}]
\PYG{k+kn}{from} \PYG{n+nn}{django}\PYG{n+nn}{.}\PYG{n+nn}{db} \PYG{k+kn}{import} \PYG{n}{models}

\PYG{k}{class} \PYG{n+nc}{Page}\PYG{p}{(}\PYG{n}{models}\PYG{o}{.}\PYG{n}{Model}\PYG{p}{)}\PYG{p}{:}
    \PYG{n}{title} \PYG{o}{=} \PYG{n}{models}\PYG{o}{.}\PYG{n}{CharField}\PYG{p}{(}\PYG{n}{max\PYGZus{}length}\PYG{o}{=}\PYG{l+m+mi}{255}\PYG{p}{)}
    \PYG{n}{url\PYGZus{}path} \PYG{o}{=} \PYG{n}{models}\PYG{o}{.}\PYG{n}{CharField}\PYG{p}{(}\PYG{n}{max\PYGZus{}length}\PYG{o}{=}\PYG{l+m+mi}{2048}\PYG{p}{)}
    \PYG{n}{language} \PYG{o}{=} \PYG{n}{models}\PYG{o}{.}\PYG{n}{CharField}\PYG{p}{(}
        \PYG{n}{max\PYGZus{}length}\PYG{o}{=}\PYG{l+m+mi}{2}\PYG{p}{,}
        \PYG{n}{choices}\PYG{o}{=}\PYG{p}{[}\PYG{p}{(}\PYG{l+s+s1}{\PYGZsq{}}\PYG{l+s+s1}{EN}\PYG{l+s+s1}{\PYGZsq{}}\PYG{p}{,} \PYG{l+s+s1}{\PYGZsq{}}\PYG{l+s+s1}{English}\PYG{l+s+s1}{\PYGZsq{}}\PYG{p}{)}\PYG{p}{,} \PYG{p}{(}\PYG{l+s+s1}{\PYGZsq{}}\PYG{l+s+s1}{DE}\PYG{l+s+s1}{\PYGZsq{}}\PYG{p}{,} \PYG{l+s+s1}{\PYGZsq{}}\PYG{l+s+s1}{Deutsch}\PYG{l+s+s1}{\PYGZsq{}}\PYG{p}{)}\PYG{p}{]}\PYG{p}{,}
        \PYG{n}{default}\PYG{o}{=}\PYG{l+s+s1}{\PYGZsq{}}\PYG{l+s+s1}{DE}\PYG{l+s+s1}{\PYGZsq{}}\PYG{p}{,}
    \PYG{p}{)}
    \PYG{n}{meta\PYGZus{}description} \PYG{o}{=} \PYG{n}{models}\PYG{o}{.}\PYG{n}{TextField}\PYG{p}{(}\PYG{n}{blank}\PYG{o}{=}\PYG{k+kc}{True}\PYG{p}{,} \PYG{n}{null}\PYG{o}{=}\PYG{k+kc}{True}\PYG{p}{)}
    \PYG{n}{meta\PYGZus{}keywords} \PYG{o}{=} \PYG{n}{models}\PYG{o}{.}\PYG{n}{CharField}\PYG{p}{(}\PYG{n}{max\PYGZus{}length}\PYG{o}{=}\PYG{l+m+mi}{255}\PYG{p}{,} \PYG{n}{blank}\PYG{o}{=}\PYG{k+kc}{True}\PYG{p}{,} \PYG{n}{null}\PYG{o}{=}\PYG{k+kc}{True}\PYG{p}{)}
    \PYG{n}{is\PYGZus{}published} \PYG{o}{=} \PYG{n}{models}\PYG{o}{.}\PYG{n}{BooleanField}\PYG{p}{(}\PYG{n}{default}\PYG{o}{=}\PYG{k+kc}{False}\PYG{p}{)}
    \PYG{n}{published\PYGZus{}at} \PYG{o}{=} \PYG{n}{models}\PYG{o}{.}\PYG{n}{DateTimeField}\PYG{p}{(}\PYG{n}{blank}\PYG{o}{=}\PYG{k+kc}{True}\PYG{p}{,} \PYG{n}{null}\PYG{o}{=}\PYG{k+kc}{True}\PYG{p}{)}
    \PYG{n}{created\PYGZus{}at} \PYG{o}{=} \PYG{n}{models}\PYG{o}{.}\PYG{n}{DateTimeField}\PYG{p}{(}\PYG{n}{auto\PYGZus{}now\PYGZus{}add}\PYG{o}{=}\PYG{k+kc}{True}\PYG{p}{)}
    \PYG{n}{updated\PYGZus{}at} \PYG{o}{=} \PYG{n}{models}\PYG{o}{.}\PYG{n}{DateTimeField}\PYG{p}{(}\PYG{n}{auto\PYGZus{}now}\PYG{o}{=}\PYG{k+kc}{True}\PYG{p}{)}

\PYG{k}{class} \PYG{n+nc}{Block}\PYG{p}{(}\PYG{n}{models}\PYG{o}{.}\PYG{n}{Model}\PYG{p}{)}\PYG{p}{:}
    \PYG{n}{template} \PYG{o}{=} \PYG{n}{models}\PYG{o}{.}\PYG{n}{TextField}\PYG{p}{(}\PYG{p}{)}
    \PYG{n}{sorting} \PYG{o}{=} \PYG{n}{models}\PYG{o}{.}\PYG{n}{IntegerField}\PYG{p}{(}\PYG{p}{)}
    \PYG{n}{created\PYGZus{}at} \PYG{o}{=} \PYG{n}{models}\PYG{o}{.}\PYG{n}{DateTimeField}\PYG{p}{(}\PYG{n}{auto\PYGZus{}now\PYGZus{}add}\PYG{o}{=}\PYG{k+kc}{True}\PYG{p}{)}
    \PYG{n}{updated\PYGZus{}at} \PYG{o}{=} \PYG{n}{models}\PYG{o}{.}\PYG{n}{DateTimeField}\PYG{p}{(}\PYG{n}{auto\PYGZus{}now}\PYG{o}{=}\PYG{k+kc}{True}\PYG{p}{)}

\PYG{k}{class} \PYG{n+nc}{MenuItem}\PYG{p}{(}\PYG{n}{models}\PYG{o}{.}\PYG{n}{Model}\PYG{p}{)}\PYG{p}{:}
    \PYG{n}{page} \PYG{o}{=} \PYG{n}{models}\PYG{o}{.}\PYG{n}{ForeignKey}\PYG{p}{(}\PYG{l+s+s1}{\PYGZsq{}}\PYG{l+s+s1}{Page}\PYG{l+s+s1}{\PYGZsq{}}\PYG{p}{,} \PYG{n}{on\PYGZus{}delete}\PYG{o}{=}\PYG{n}{models}\PYG{o}{.}\PYG{n}{CASCADE}\PYG{p}{)}
    \PYG{n}{template} \PYG{o}{=} \PYG{n}{models}\PYG{o}{.}\PYG{n}{TextField}\PYG{p}{(}\PYG{p}{)}
    \PYG{n}{sorting} \PYG{o}{=} \PYG{n}{models}\PYG{o}{.}\PYG{n}{IntegerField}\PYG{p}{(}\PYG{p}{)}
    \PYG{n}{created\PYGZus{}at} \PYG{o}{=} \PYG{n}{models}\PYG{o}{.}\PYG{n}{DateTimeField}\PYG{p}{(}\PYG{n}{auto\PYGZus{}now\PYGZus{}add}\PYG{o}{=}\PYG{k+kc}{True}\PYG{p}{)}
    \PYG{n}{updated\PYGZus{}at} \PYG{o}{=} \PYG{n}{models}\PYG{o}{.}\PYG{n}{DateTimeField}\PYG{p}{(}\PYG{n}{auto\PYGZus{}now}\PYG{o}{=}\PYG{k+kc}{True}\PYG{p}{)}

\PYG{k}{class} \PYG{n+nc}{PageBlock}\PYG{p}{(}\PYG{n}{models}\PYG{o}{.}\PYG{n}{Model}\PYG{p}{)}\PYG{p}{:}
    \PYG{n}{page} \PYG{o}{=} \PYG{n}{models}\PYG{o}{.}\PYG{n}{ForeignKey}\PYG{p}{(}\PYG{l+s+s1}{\PYGZsq{}}\PYG{l+s+s1}{Page}\PYG{l+s+s1}{\PYGZsq{}}\PYG{p}{,} \PYG{n}{on\PYGZus{}delete}\PYG{o}{=}\PYG{n}{models}\PYG{o}{.}\PYG{n}{CASCADE}\PYG{p}{)}
    \PYG{n}{block} \PYG{o}{=} \PYG{n}{models}\PYG{o}{.}\PYG{n}{ForeignKey}\PYG{p}{(}\PYG{l+s+s1}{\PYGZsq{}}\PYG{l+s+s1}{Block}\PYG{l+s+s1}{\PYGZsq{}}\PYG{p}{,} \PYG{n}{on\PYGZus{}delete}\PYG{o}{=}\PYG{n}{models}\PYG{o}{.}\PYG{n}{CASCADE}\PYG{p}{)}
\end{sphinxVerbatim}


\subsection{Cache\sphinxhyphen{}Strategie}
\label{\detokenize{sections/entwurf:cache-strategie}}
\sphinxAtStartPar
Das Backend implementiert eine effiziente Cache\sphinxhyphen{}Strategie, um die Performance zu optimieren. Es wird Redis für das Caching verwendet, um schnelle Zugriffszeiten zu gewährleisten.

\begin{sphinxVerbatim}[commandchars=\\\{\}]
\PYG{k+kn}{from} \PYG{n+nn}{django}\PYG{n+nn}{.}\PYG{n+nn}{core}\PYG{n+nn}{.}\PYG{n+nn}{cache} \PYG{k+kn}{import} \PYG{n}{cache}
\PYG{k+kn}{from} \PYG{n+nn}{django}\PYG{n+nn}{.}\PYG{n+nn}{db}\PYG{n+nn}{.}\PYG{n+nn}{models}\PYG{n+nn}{.}\PYG{n+nn}{signals} \PYG{k+kn}{import} \PYG{n}{post\PYGZus{}save}\PYG{p}{,} \PYG{n}{post\PYGZus{}delete}
\PYG{k+kn}{from} \PYG{n+nn}{django}\PYG{n+nn}{.}\PYG{n+nn}{dispatch} \PYG{k+kn}{import} \PYG{n}{receiver}

\PYG{n+nd}{@receiver}\PYG{p}{(}\PYG{p}{[}\PYG{n}{post\PYGZus{}save}\PYG{p}{,} \PYG{n}{post\PYGZus{}delete}\PYG{p}{]}\PYG{p}{,} \PYG{n}{sender}\PYG{o}{=}\PYG{n}{Block}\PYG{p}{)}
\PYG{n+nd}{@receiver}\PYG{p}{(}\PYG{p}{[}\PYG{n}{post\PYGZus{}save}\PYG{p}{,} \PYG{n}{post\PYGZus{}delete}\PYG{p}{]}\PYG{p}{,} \PYG{n}{sender}\PYG{o}{=}\PYG{n}{MenuItem}\PYG{p}{)}
\PYG{n+nd}{@receiver}\PYG{p}{(}\PYG{p}{[}\PYG{n}{post\PYGZus{}save}\PYG{p}{,} \PYG{n}{post\PYGZus{}delete}\PYG{p}{]}\PYG{p}{,} \PYG{n}{sender}\PYG{o}{=}\PYG{n}{PageBlock}\PYG{p}{)}
\PYG{k}{def} \PYG{n+nf}{invalidate\PYGZus{}cache}\PYG{p}{(}\PYG{n}{sender}\PYG{p}{,} \PYG{o}{*}\PYG{o}{*}\PYG{n}{kwargs}\PYG{p}{)}\PYG{p}{:}
    \PYG{n}{cache}\PYG{o}{.}\PYG{n}{clear}\PYG{p}{(}\PYG{p}{)}
\end{sphinxVerbatim}

\sphinxstepscope


\section{Implementierung}
\label{\detokenize{sections/implementierung:implementierung}}\label{\detokenize{sections/implementierung::doc}}

\subsection{Details der Umsetzung}
\label{\detokenize{sections/implementierung:details-der-umsetzung}}
\sphinxAtStartPar
Die Implementierung des Backends umfasst mehrere Kernkomponenten und Technologien, die im Folgenden beschrieben werden.


\subsection{Page\sphinxhyphen{}Modell}
\label{\detokenize{sections/implementierung:page-modell}}
\sphinxAtStartPar
Das Page\sphinxhyphen{}Modell ist die zentrale Entität für Webseiteninhalte. Es unterstützt mehrsprachige Inhalte und verwaltet Meta\sphinxhyphen{}Informationen.

\begin{sphinxVerbatim}[commandchars=\\\{\}]
\PYG{k+kn}{from} \PYG{n+nn}{django}\PYG{n+nn}{.}\PYG{n+nn}{db} \PYG{k+kn}{import} \PYG{n}{models}

\PYG{k}{class} \PYG{n+nc}{Page}\PYG{p}{(}\PYG{n}{models}\PYG{o}{.}\PYG{n}{Model}\PYG{p}{)}\PYG{p}{:}
    \PYG{n}{title} \PYG{o}{=} \PYG{n}{models}\PYG{o}{.}\PYG{n}{CharField}\PYG{p}{(}\PYG{n}{max\PYGZus{}length}\PYG{o}{=}\PYG{l+m+mi}{255}\PYG{p}{)}
    \PYG{n}{url\PYGZus{}path} \PYG{o}{=} \PYG{n}{models}\PYG{o}{.}\PYG{n}{CharField}\PYG{p}{(}\PYG{n}{max\PYGZus{}length}\PYG{o}{=}\PYG{l+m+mi}{2048}\PYG{p}{)}
    \PYG{n}{language} \PYG{o}{=} \PYG{n}{models}\PYG{o}{.}\PYG{n}{CharField}\PYG{p}{(}
        \PYG{n}{max\PYGZus{}length}\PYG{o}{=}\PYG{l+m+mi}{2}\PYG{p}{,}
        \PYG{n}{choices}\PYG{o}{=}\PYG{p}{[}\PYG{p}{(}\PYG{l+s+s1}{\PYGZsq{}}\PYG{l+s+s1}{EN}\PYG{l+s+s1}{\PYGZsq{}}\PYG{p}{,} \PYG{l+s+s1}{\PYGZsq{}}\PYG{l+s+s1}{English}\PYG{l+s+s1}{\PYGZsq{}}\PYG{p}{)}\PYG{p}{,} \PYG{p}{(}\PYG{l+s+s1}{\PYGZsq{}}\PYG{l+s+s1}{DE}\PYG{l+s+s1}{\PYGZsq{}}\PYG{p}{,} \PYG{l+s+s1}{\PYGZsq{}}\PYG{l+s+s1}{Deutsch}\PYG{l+s+s1}{\PYGZsq{}}\PYG{p}{)}\PYG{p}{]}\PYG{p}{,}
        \PYG{n}{default}\PYG{o}{=}\PYG{l+s+s1}{\PYGZsq{}}\PYG{l+s+s1}{DE}\PYG{l+s+s1}{\PYGZsq{}}\PYG{p}{,}
    \PYG{p}{)}
    \PYG{n}{meta\PYGZus{}description} \PYG{o}{=} \PYG{n}{models}\PYG{o}{.}\PYG{n}{TextField}\PYG{p}{(}\PYG{n}{blank}\PYG{o}{=}\PYG{k+kc}{True}\PYG{p}{,} \PYG{n}{null}\PYG{o}{=}\PYG{k+kc}{True}\PYG{p}{)}
    \PYG{n}{meta\PYGZus{}keywords} \PYG{o}{=} \PYG{n}{models}\PYG{o}{.}\PYG{n}{CharField}\PYG{p}{(}\PYG{n}{max\PYGZus{}length}\PYG{o}{=}\PYG{l+m+mi}{255}\PYG{p}{,} \PYG{n}{blank}\PYG{o}{=}\PYG{k+kc}{True}\PYG{p}{,} \PYG{n}{null}\PYG{o}{=}\PYG{k+kc}{True}\PYG{p}{)}
    \PYG{n}{is\PYGZus{}published} \PYG{o}{=} \PYG{n}{models}\PYG{o}{.}\PYG{n}{BooleanField}\PYG{p}{(}\PYG{n}{default}\PYG{o}{=}\PYG{k+kc}{False}\PYG{p}{)}
    \PYG{n}{published\PYGZus{}at} \PYG{o}{=} \PYG{n}{models}\PYG{o}{.}\PYG{n}{DateTimeField}\PYG{p}{(}\PYG{n}{blank}\PYG{o}{=}\PYG{k+kc}{True}\PYG{p}{,} \PYG{n}{null}\PYG{o}{=}\PYG{k+kc}{True}\PYG{p}{)}
    \PYG{n}{created\PYGZus{}at} \PYG{o}{=} \PYG{n}{models}\PYG{o}{.}\PYG{n}{DateTimeField}\PYG{p}{(}\PYG{n}{auto\PYGZus{}now\PYGZus{}add}\PYG{o}{=}\PYG{k+kc}{True}\PYG{p}{)}
    \PYG{n}{updated\PYGZus{}at} \PYG{o}{=} \PYG{n}{models}\PYG{o}{.}\PYG{n}{DateTimeField}\PYG{p}{(}\PYG{n}{auto\PYGZus{}now}\PYG{o}{=}\PYG{k+kc}{True}\PYG{p}{)}
\end{sphinxVerbatim}


\subsection{Block\sphinxhyphen{}Modell}
\label{\detokenize{sections/implementierung:block-modell}}
\sphinxAtStartPar
Das Block\sphinxhyphen{}Modell repräsentiert wiederverwendbare Inhaltskomponenten, die in mehreren Seiten verwendet werden können.

\begin{sphinxVerbatim}[commandchars=\\\{\}]
\PYG{k+kn}{from} \PYG{n+nn}{django}\PYG{n+nn}{.}\PYG{n+nn}{db} \PYG{k+kn}{import} \PYG{n}{models}

\PYG{k}{class} \PYG{n+nc}{Block}\PYG{p}{(}\PYG{n}{models}\PYG{o}{.}\PYG{n}{Model}\PYG{p}{)}\PYG{p}{:}
    \PYG{n}{template} \PYG{o}{=} \PYG{n}{models}\PYG{o}{.}\PYG{n}{TextField}\PYG{p}{(}\PYG{p}{)}
    \PYG{n}{sorting} \PYG{o}{=} \PYG{n}{models}\PYG{o}{.}\PYG{n}{IntegerField}\PYG{p}{(}\PYG{p}{)}
    \PYG{n}{created\PYGZus{}at} \PYG{o}{=} \PYG{n}{models}\PYG{o}{.}\PYG{n}{DateTimeField}\PYG{p}{(}\PYG{n}{auto\PYGZus{}now\PYGZus{}add}\PYG{o}{=}\PYG{k+kc}{True}\PYG{p}{)}
    \PYG{n}{updated\PYGZus{}at} \PYG{o}{=} \PYG{n}{models}\PYG{o}{.}\PYG{n}{DateTimeField}\PYG{p}{(}\PYG{n}{auto\PYGZus{}now}\PYG{o}{=}\PYG{k+kc}{True}\PYG{p}{)}
\end{sphinxVerbatim}


\subsection{MenuItem\sphinxhyphen{}Modell}
\label{\detokenize{sections/implementierung:menuitem-modell}}
\sphinxAtStartPar
Das MenuItem\sphinxhyphen{}Modell verwaltet Navigationsstrukturen und verlinkt Seiten mit anpassbaren Templates.

\begin{sphinxVerbatim}[commandchars=\\\{\}]
\PYG{k+kn}{from} \PYG{n+nn}{django}\PYG{n+nn}{.}\PYG{n+nn}{db} \PYG{k+kn}{import} \PYG{n}{models}

\PYG{k}{class} \PYG{n+nc}{MenuItem}\PYG{p}{(}\PYG{n}{models}\PYG{o}{.}\PYG{n}{Model}\PYG{p}{)}\PYG{p}{:}
    \PYG{n}{page} \PYG{o}{=} \PYG{n}{models}\PYG{o}{.}\PYG{n}{ForeignKey}\PYG{p}{(}\PYG{l+s+s1}{\PYGZsq{}}\PYG{l+s+s1}{Page}\PYG{l+s+s1}{\PYGZsq{}}\PYG{p}{,} \PYG{n}{on\PYGZus{}delete}\PYG{o}{=}\PYG{n}{models}\PYG{o}{.}\PYG{n}{CASCADE}\PYG{p}{)}
    \PYG{n}{template} \PYG{o}{=} \PYG{n}{models}\PYG{o}{.}\PYG{n}{TextField}\PYG{p}{(}\PYG{p}{)}
    \PYG{n}{sorting} \PYG{o}{=} \PYG{n}{models}\PYG{o}{.}\PYG{n}{IntegerField}\PYG{p}{(}\PYG{p}{)}
    \PYG{n}{created\PYGZus{}at} \PYG{o}{=} \PYG{n}{models}\PYG{o}{.}\PYG{n}{DateTimeField}\PYG{p}{(}\PYG{n}{auto\PYGZus{}now\PYGZus{}add}\PYG{o}{=}\PYG{k+kc}{True}\PYG{p}{)}
    \PYG{n}{updated\PYGZus{}at} \PYG{o}{=} \PYG{n}{models}\PYG{o}{.}\PYG{n}{DateTimeField}\PYG{p}{(}\PYG{n}{auto\PYGZus{}now}\PYG{o}{=}\PYG{k+kc}{True}\PYG{p}{)}
\end{sphinxVerbatim}


\subsection{PageBlock\sphinxhyphen{}Modell}
\label{\detokenize{sections/implementierung:pageblock-modell}}
\sphinxAtStartPar
Das PageBlock\sphinxhyphen{}Modell verwaltet die Beziehungen zwischen Seiten und Blöcken.

\begin{sphinxVerbatim}[commandchars=\\\{\}]
\PYG{k+kn}{from} \PYG{n+nn}{django}\PYG{n+nn}{.}\PYG{n+nn}{db} \PYG{k+kn}{import} \PYG{n}{models}

\PYG{k}{class} \PYG{n+nc}{PageBlock}\PYG{p}{(}\PYG{n}{models}\PYG{o}{.}\PYG{n}{Model}\PYG{p}{)}\PYG{p}{:}
    \PYG{n}{page} \PYG{o}{=} \PYG{n}{models}\PYG{o}{.}\PYG{n}{ForeignKey}\PYG{p}{(}\PYG{l+s+s1}{\PYGZsq{}}\PYG{l+s+s1}{Page}\PYG{l+s+s1}{\PYGZsq{}}\PYG{p}{,} \PYG{n}{on\PYGZus{}delete}\PYG{o}{=}\PYG{n}{models}\PYG{o}{.}\PYG{n}{CASCADE}\PYG{p}{)}
    \PYG{n}{block} \PYG{o}{=} \PYG{n}{models}\PYG{o}{.}\PYG{n}{ForeignKey}\PYG{p}{(}\PYG{l+s+s1}{\PYGZsq{}}\PYG{l+s+s1}{Block}\PYG{l+s+s1}{\PYGZsq{}}\PYG{p}{,} \PYG{n}{on\PYGZus{}delete}\PYG{o}{=}\PYG{n}{models}\PYG{o}{.}\PYG{n}{CASCADE}\PYG{p}{)}
\end{sphinxVerbatim}


\subsection{Caching}
\label{\detokenize{sections/implementierung:caching}}
\sphinxAtStartPar
Das Backend implementiert eine effiziente Cache\sphinxhyphen{}Strategie, um die Performance zu optimieren. Es wird Redis für das Caching verwendet.

\begin{sphinxVerbatim}[commandchars=\\\{\}]
\PYG{k+kn}{from} \PYG{n+nn}{django}\PYG{n+nn}{.}\PYG{n+nn}{core}\PYG{n+nn}{.}\PYG{n+nn}{cache} \PYG{k+kn}{import} \PYG{n}{cache}
\PYG{k+kn}{from} \PYG{n+nn}{django}\PYG{n+nn}{.}\PYG{n+nn}{db}\PYG{n+nn}{.}\PYG{n+nn}{models}\PYG{n+nn}{.}\PYG{n+nn}{signals} \PYG{k+kn}{import} \PYG{n}{post\PYGZus{}save}\PYG{p}{,} \PYG{n}{post\PYGZus{}delete}
\PYG{k+kn}{from} \PYG{n+nn}{django}\PYG{n+nn}{.}\PYG{n+nn}{dispatch} \PYG{k+kn}{import} \PYG{n}{receiver}

\PYG{n+nd}{@receiver}\PYG{p}{(}\PYG{p}{[}\PYG{n}{post\PYGZus{}save}\PYG{p}{,} \PYG{n}{post\PYGZus{}delete}\PYG{p}{]}\PYG{p}{,} \PYG{n}{sender}\PYG{o}{=}\PYG{n}{Block}\PYG{p}{)}
\PYG{n+nd}{@receiver}\PYG{p}{(}\PYG{p}{[}\PYG{n}{post\PYGZus{}save}\PYG{p}{,} \PYG{n}{post\PYGZus{}delete}\PYG{p}{]}\PYG{p}{,} \PYG{n}{sender}\PYG{o}{=}\PYG{n}{MenuItem}\PYG{p}{)}
\PYG{n+nd}{@receiver}\PYG{p}{(}\PYG{p}{[}\PYG{n}{post\PYGZus{}save}\PYG{p}{,} \PYG{n}{post\PYGZus{}delete}\PYG{p}{]}\PYG{p}{,} \PYG{n}{sender}\PYG{o}{=}\PYG{n}{PageBlock}\PYG{p}{)}
\PYG{k}{def} \PYG{n+nf}{invalidate\PYGZus{}cache}\PYG{p}{(}\PYG{n}{sender}\PYG{p}{,} \PYG{o}{*}\PYG{o}{*}\PYG{n}{kwargs}\PYG{p}{)}\PYG{p}{:}
    \PYG{n}{cache}\PYG{o}{.}\PYG{n}{clear}\PYG{p}{(}\PYG{p}{)}
\end{sphinxVerbatim}


\subsection{Django Signals}
\label{\detokenize{sections/implementierung:django-signals}}
\sphinxAtStartPar
Django Signals werden verwendet, um bestimmte Aktionen automatisch auszulösen, wenn Änderungen an den Modellen vorgenommen werden.

\begin{sphinxVerbatim}[commandchars=\\\{\}]
\PYG{k+kn}{from} \PYG{n+nn}{django}\PYG{n+nn}{.}\PYG{n+nn}{db}\PYG{n+nn}{.}\PYG{n+nn}{models}\PYG{n+nn}{.}\PYG{n+nn}{signals} \PYG{k+kn}{import} \PYG{n}{post\PYGZus{}save}
\PYG{k+kn}{from} \PYG{n+nn}{django}\PYG{n+nn}{.}\PYG{n+nn}{dispatch} \PYG{k+kn}{import} \PYG{n}{receiver}
\PYG{k+kn}{from} \PYG{n+nn}{.}\PYG{n+nn}{models} \PYG{k+kn}{import} \PYG{n}{Page}

\PYG{n+nd}{@receiver}\PYG{p}{(}\PYG{n}{post\PYGZus{}save}\PYG{p}{,} \PYG{n}{sender}\PYG{o}{=}\PYG{n}{Page}\PYG{p}{)}
\PYG{k}{def} \PYG{n+nf}{update\PYGZus{}page\PYGZus{}cache}\PYG{p}{(}\PYG{n}{sender}\PYG{p}{,} \PYG{n}{instance}\PYG{p}{,} \PYG{o}{*}\PYG{o}{*}\PYG{n}{kwargs}\PYG{p}{)}\PYG{p}{:}
    \PYG{n}{cache\PYGZus{}key} \PYG{o}{=} \PYG{l+s+sa}{f}\PYG{l+s+s2}{\PYGZdq{}}\PYG{l+s+s2}{page\PYGZus{}}\PYG{l+s+si}{\PYGZob{}}\PYG{n}{instance}\PYG{o}{.}\PYG{n}{url\PYGZus{}path}\PYG{l+s+si}{\PYGZcb{}}\PYG{l+s+s2}{\PYGZus{}}\PYG{l+s+si}{\PYGZob{}}\PYG{n}{instance}\PYG{o}{.}\PYG{n}{language}\PYG{l+s+si}{\PYGZcb{}}\PYG{l+s+s2}{\PYGZdq{}}
    \PYG{n}{cache}\PYG{o}{.}\PYG{n}{set}\PYG{p}{(}\PYG{n}{cache\PYGZus{}key}\PYG{p}{,} \PYG{n}{instance}\PYG{p}{)}
\end{sphinxVerbatim}


\subsection{Software \& Technologien}
\label{\detokenize{sections/implementierung:software-technologien}}\begin{itemize}
\item {} \begin{description}
\sphinxlineitem{\sphinxstylestrong{Django\sphinxhyphen{}Ökosystem:}}\begin{itemize}
\item {} 
\sphinxAtStartPar
Django Framework

\item {} 
\sphinxAtStartPar
django\sphinxhyphen{}simple\sphinxhyphen{}history

\item {} 
\sphinxAtStartPar
django\sphinxhyphen{}redis

\item {} 
\sphinxAtStartPar
django\sphinxhyphen{}environ

\item {} 
\sphinxAtStartPar
Django Debug Toolbar

\end{itemize}

\end{description}

\item {} \begin{description}
\sphinxlineitem{\sphinxstylestrong{Datenbank \& Caching:}}\begin{itemize}
\item {} 
\sphinxAtStartPar
SQLite (Entwicklung)

\item {} 
\sphinxAtStartPar
PostgreSQL (Produktion)

\item {} 
\sphinxAtStartPar
Redis (Caching\sphinxhyphen{}Server)

\end{itemize}

\end{description}

\item {} \begin{description}
\sphinxlineitem{\sphinxstylestrong{Server \& Deployment:}}\begin{itemize}
\item {} 
\sphinxAtStartPar
Gunicorn (WSGI)

\item {} 
\sphinxAtStartPar
Nginx

\item {} 
\sphinxAtStartPar
WhiteNoise

\item {} 
\sphinxAtStartPar
systemd

\end{itemize}

\end{description}

\item {} \begin{description}
\sphinxlineitem{\sphinxstylestrong{Testing \& QA:}}\begin{itemize}
\item {} 
\sphinxAtStartPar
PyTest \& PyTest\sphinxhyphen{}Django

\item {} 
\sphinxAtStartPar
FactoryBoy

\item {} 
\sphinxAtStartPar
Coverage.py

\end{itemize}

\end{description}

\item {} \begin{description}
\sphinxlineitem{\sphinxstylestrong{Dokumentation:}}\begin{itemize}
\item {} 
\sphinxAtStartPar
Sphinx

\item {} 
\sphinxAtStartPar
sphinx\sphinxhyphen{}rtd\sphinxhyphen{}theme

\item {} 
\sphinxAtStartPar
sphinxcontrib\sphinxhyphen{}plantuml

\end{itemize}

\end{description}

\item {} \begin{description}
\sphinxlineitem{\sphinxstylestrong{Entwicklungstools:}}\begin{itemize}
\item {} 
\sphinxAtStartPar
Python 3.8+

\item {} 
\sphinxAtStartPar
PyCharm IDE

\item {} 
\sphinxAtStartPar
Git \& GitLab

\item {} 
\sphinxAtStartPar
make

\end{itemize}

\end{description}

\item {} \begin{description}
\sphinxlineitem{\sphinxstylestrong{Media \& Assets:}}\begin{itemize}
\item {} 
\sphinxAtStartPar
Pillow

\end{itemize}

\end{description}

\end{itemize}

\sphinxstepscope


\section{Test und Durchführung}
\label{\detokenize{sections/test_durchfuehrung:test-und-durchfuhrung}}\label{\detokenize{sections/test_durchfuehrung::doc}}
\sphinxAtStartPar
Diese Sektion umfasst die implementierten Tests und ihre Ergebnisse.


\subsection{Testphase}
\label{\detokenize{sections/test_durchfuehrung:testphase}}
\sphinxAtStartPar
Die Tests stellen sicher, dass die Kernfunktionalitäten des CMS korrekt implementiert sind:
\begin{itemize}
\item {} 
\sphinxAtStartPar
\sphinxstylestrong{Model Tests}: Validierung der Datenmodelle und ihrer Beziehungen

\item {} 
\sphinxAtStartPar
\sphinxstylestrong{Cache Tests}: Überprüfung der Redis\sphinxhyphen{}Cache Implementierung

\item {} 
\sphinxAtStartPar
\sphinxstylestrong{View Tests}: Tests der Seiten\sphinxhyphen{}Rendering und URL\sphinxhyphen{}Routing

\end{itemize}


\subsection{Implementierte Tests}
\label{\detokenize{sections/test_durchfuehrung:implementierte-tests}}

\subsubsection{Model Tests (test\_models.py)}
\label{\detokenize{sections/test_durchfuehrung:model-tests-test-models-py}}
\begin{sphinxVerbatim}[commandchars=\\\{\}]
\PYG{k}{class} \PYG{n+nc}{PageModelTest}\PYG{p}{(}\PYG{n}{TestCase}\PYG{p}{)}\PYG{p}{:}
    \PYG{k}{def} \PYG{n+nf}{test\PYGZus{}page\PYGZus{}creation}\PYG{p}{(}\PYG{n+nb+bp}{self}\PYG{p}{)}\PYG{p}{:}
\PYG{+w}{        }\PYG{l+s+sd}{\PYGZdq{}\PYGZdq{}\PYGZdq{}Test page creation with all fields.\PYGZdq{}\PYGZdq{}\PYGZdq{}}
        \PYG{n}{page} \PYG{o}{=} \PYG{n}{Page}\PYG{o}{.}\PYG{n}{objects}\PYG{o}{.}\PYG{n}{create}\PYG{p}{(}
            \PYG{n}{title}\PYG{o}{=}\PYG{l+s+s2}{\PYGZdq{}}\PYG{l+s+s2}{Test Page}\PYG{l+s+s2}{\PYGZdq{}}\PYG{p}{,}
            \PYG{n}{url\PYGZus{}path}\PYG{o}{=}\PYG{l+s+s2}{\PYGZdq{}}\PYG{l+s+s2}{/test}\PYG{l+s+s2}{\PYGZdq{}}\PYG{p}{,}
            \PYG{n}{language}\PYG{o}{=}\PYG{l+s+s2}{\PYGZdq{}}\PYG{l+s+s2}{EN}\PYG{l+s+s2}{\PYGZdq{}}
        \PYG{p}{)}
        \PYG{n+nb+bp}{self}\PYG{o}{.}\PYG{n}{assertEqual}\PYG{p}{(}\PYG{n}{page}\PYG{o}{.}\PYG{n}{title}\PYG{p}{,} \PYG{l+s+s2}{\PYGZdq{}}\PYG{l+s+s2}{Test Page}\PYG{l+s+s2}{\PYGZdq{}}\PYG{p}{)}

\PYG{k}{class} \PYG{n+nc}{BlockModelTest}\PYG{p}{(}\PYG{n}{TestCase}\PYG{p}{)}\PYG{p}{:}
    \PYG{k}{def} \PYG{n+nf}{test\PYGZus{}block\PYGZus{}creation}\PYG{p}{(}\PYG{n+nb+bp}{self}\PYG{p}{)}\PYG{p}{:}
\PYG{+w}{        }\PYG{l+s+sd}{\PYGZdq{}\PYGZdq{}\PYGZdq{}Test block creation with all fields.\PYGZdq{}\PYGZdq{}\PYGZdq{}}
        \PYG{n}{block} \PYG{o}{=} \PYG{n}{Block}\PYG{o}{.}\PYG{n}{objects}\PYG{o}{.}\PYG{n}{create}\PYG{p}{(}
            \PYG{n}{name}\PYG{o}{=}\PYG{l+s+s2}{\PYGZdq{}}\PYG{l+s+s2}{Test Block}\PYG{l+s+s2}{\PYGZdq{}}\PYG{p}{,}
            \PYG{n}{template}\PYG{o}{=}\PYG{l+s+s2}{\PYGZdq{}}\PYG{l+s+s2}{\PYGZlt{}div\PYGZgt{}Test Content\PYGZlt{}/div\PYGZgt{}}\PYG{l+s+s2}{\PYGZdq{}}\PYG{p}{,}
            \PYG{n}{sorting}\PYG{o}{=}\PYG{l+m+mi}{1}
        \PYG{p}{)}
        \PYG{n+nb+bp}{self}\PYG{o}{.}\PYG{n}{assertEqual}\PYG{p}{(}\PYG{n+nb}{str}\PYG{p}{(}\PYG{n}{block}\PYG{p}{)}\PYG{p}{,} \PYG{l+s+s2}{\PYGZdq{}}\PYG{l+s+s2}{Test Block}\PYG{l+s+s2}{\PYGZdq{}}\PYG{p}{)}

\PYG{k}{class} \PYG{n+nc}{MenuItemModelTest}\PYG{p}{(}\PYG{n}{TestCase}\PYG{p}{)}\PYG{p}{:}
    \PYG{k}{def} \PYG{n+nf}{test\PYGZus{}menu\PYGZus{}item\PYGZus{}ordering}\PYG{p}{(}\PYG{n+nb+bp}{self}\PYG{p}{)}\PYG{p}{:}
\PYG{+w}{        }\PYG{l+s+sd}{\PYGZdq{}\PYGZdq{}\PYGZdq{}Test menu item ordering by sorting field.\PYGZdq{}\PYGZdq{}\PYGZdq{}}
        \PYG{n}{menu1} \PYG{o}{=} \PYG{n}{MenuItem}\PYG{o}{.}\PYG{n}{objects}\PYG{o}{.}\PYG{n}{create}\PYG{p}{(}\PYG{n}{page}\PYG{o}{=}\PYG{n+nb+bp}{self}\PYG{o}{.}\PYG{n}{page}\PYG{p}{,} \PYG{n}{template}\PYG{o}{=}\PYG{l+s+s2}{\PYGZdq{}}\PYG{l+s+s2}{test1}\PYG{l+s+s2}{\PYGZdq{}}\PYG{p}{,} \PYG{n}{sorting}\PYG{o}{=}\PYG{l+m+mi}{2}\PYG{p}{)}
        \PYG{n}{menu2} \PYG{o}{=} \PYG{n}{MenuItem}\PYG{o}{.}\PYG{n}{objects}\PYG{o}{.}\PYG{n}{create}\PYG{p}{(}\PYG{n}{page}\PYG{o}{=}\PYG{n+nb+bp}{self}\PYG{o}{.}\PYG{n}{page}\PYG{p}{,} \PYG{n}{template}\PYG{o}{=}\PYG{l+s+s2}{\PYGZdq{}}\PYG{l+s+s2}{test2}\PYG{l+s+s2}{\PYGZdq{}}\PYG{p}{,} \PYG{n}{sorting}\PYG{o}{=}\PYG{l+m+mi}{1}\PYG{p}{)}
        \PYG{n}{menus} \PYG{o}{=} \PYG{n}{MenuItem}\PYG{o}{.}\PYG{n}{objects}\PYG{o}{.}\PYG{n}{all}\PYG{p}{(}\PYG{p}{)}\PYG{o}{.}\PYG{n}{order\PYGZus{}by}\PYG{p}{(}\PYG{l+s+s2}{\PYGZdq{}}\PYG{l+s+s2}{sorting}\PYG{l+s+s2}{\PYGZdq{}}\PYG{p}{)}
        \PYG{n+nb+bp}{self}\PYG{o}{.}\PYG{n}{assertEqual}\PYG{p}{(}\PYG{n}{menus}\PYG{p}{[}\PYG{l+m+mi}{0}\PYG{p}{]}\PYG{p}{,} \PYG{n}{menu2}\PYG{p}{)}
\end{sphinxVerbatim}


\subsubsection{Cache Tests (test\_cache.py)}
\label{\detokenize{sections/test_durchfuehrung:cache-tests-test-cache-py}}
\begin{sphinxVerbatim}[commandchars=\\\{\}]
\PYG{k}{class} \PYG{n+nc}{CacheTests}\PYG{p}{(}\PYG{n}{TestCase}\PYG{p}{)}\PYG{p}{:}
    \PYG{k}{def} \PYG{n+nf}{test\PYGZus{}cache\PYGZus{}set\PYGZus{}get}\PYG{p}{(}\PYG{n+nb+bp}{self}\PYG{p}{)}\PYG{p}{:}
\PYG{+w}{        }\PYG{l+s+sd}{\PYGZdq{}\PYGZdq{}\PYGZdq{}Test basic cache set and get operations.\PYGZdq{}\PYGZdq{}\PYGZdq{}}
        \PYG{n}{cache}\PYG{o}{.}\PYG{n}{set}\PYG{p}{(}\PYG{l+s+s2}{\PYGZdq{}}\PYG{l+s+s2}{test\PYGZus{}key}\PYG{l+s+s2}{\PYGZdq{}}\PYG{p}{,} \PYG{l+s+s2}{\PYGZdq{}}\PYG{l+s+s2}{test\PYGZus{}value}\PYG{l+s+s2}{\PYGZdq{}}\PYG{p}{)}
        \PYG{n+nb+bp}{self}\PYG{o}{.}\PYG{n}{assertEqual}\PYG{p}{(}\PYG{n}{cache}\PYG{o}{.}\PYG{n}{get}\PYG{p}{(}\PYG{l+s+s2}{\PYGZdq{}}\PYG{l+s+s2}{test\PYGZus{}key}\PYG{l+s+s2}{\PYGZdq{}}\PYG{p}{)}\PYG{p}{,} \PYG{l+s+s2}{\PYGZdq{}}\PYG{l+s+s2}{test\PYGZus{}value}\PYG{l+s+s2}{\PYGZdq{}}\PYG{p}{)}

    \PYG{k}{def} \PYG{n+nf}{test\PYGZus{}cache\PYGZus{}timeout}\PYG{p}{(}\PYG{n+nb+bp}{self}\PYG{p}{)}\PYG{p}{:}
\PYG{+w}{        }\PYG{l+s+sd}{\PYGZdq{}\PYGZdq{}\PYGZdq{}Test cache timeout functionality.\PYGZdq{}\PYGZdq{}\PYGZdq{}}
        \PYG{n}{cache}\PYG{o}{.}\PYG{n}{set}\PYG{p}{(}\PYG{l+s+s2}{\PYGZdq{}}\PYG{l+s+s2}{timeout\PYGZus{}key}\PYG{l+s+s2}{\PYGZdq{}}\PYG{p}{,} \PYG{l+s+s2}{\PYGZdq{}}\PYG{l+s+s2}{timeout\PYGZus{}value}\PYG{l+s+s2}{\PYGZdq{}}\PYG{p}{,} \PYG{n}{timeout}\PYG{o}{=}\PYG{l+m+mi}{1}\PYG{p}{)}
        \PYG{n}{time}\PYG{o}{.}\PYG{n}{sleep}\PYG{p}{(}\PYG{l+m+mi}{2}\PYG{p}{)}
        \PYG{n+nb+bp}{self}\PYG{o}{.}\PYG{n}{assertIsNone}\PYG{p}{(}\PYG{n}{cache}\PYG{o}{.}\PYG{n}{get}\PYG{p}{(}\PYG{l+s+s2}{\PYGZdq{}}\PYG{l+s+s2}{timeout\PYGZus{}key}\PYG{l+s+s2}{\PYGZdq{}}\PYG{p}{)}\PYG{p}{)}
\end{sphinxVerbatim}


\subsubsection{View Tests (test\_views.py)}
\label{\detokenize{sections/test_durchfuehrung:view-tests-test-views-py}}
\begin{sphinxVerbatim}[commandchars=\\\{\}]
\PYG{k}{class} \PYG{n+nc}{ViewTests}\PYG{p}{(}\PYG{n}{TestCase}\PYG{p}{)}\PYG{p}{:}
    \PYG{k}{def} \PYG{n+nf}{test\PYGZus{}home\PYGZus{}view}\PYG{p}{(}\PYG{n+nb+bp}{self}\PYG{p}{)}\PYG{p}{:}
\PYG{+w}{        }\PYG{l+s+sd}{\PYGZdq{}\PYGZdq{}\PYGZdq{}Test home page rendering.\PYGZdq{}\PYGZdq{}\PYGZdq{}}
        \PYG{n}{response} \PYG{o}{=} \PYG{n+nb+bp}{self}\PYG{o}{.}\PYG{n}{client}\PYG{o}{.}\PYG{n}{get}\PYG{p}{(}\PYG{l+s+s2}{\PYGZdq{}}\PYG{l+s+s2}{/}\PYG{l+s+s2}{\PYGZdq{}}\PYG{p}{)}
        \PYG{n+nb+bp}{self}\PYG{o}{.}\PYG{n}{assertEqual}\PYG{p}{(}\PYG{n}{response}\PYG{o}{.}\PYG{n}{status\PYGZus{}code}\PYG{p}{,} \PYG{l+m+mi}{200}\PYG{p}{)}

    \PYG{k}{def} \PYG{n+nf}{test\PYGZus{}render\PYGZus{}page\PYGZus{}existing\PYGZus{}page}\PYG{p}{(}\PYG{n+nb+bp}{self}\PYG{p}{)}\PYG{p}{:}
\PYG{+w}{        }\PYG{l+s+sd}{\PYGZdq{}\PYGZdq{}\PYGZdq{}Test rendering of an existing published page.\PYGZdq{}\PYGZdq{}\PYGZdq{}}
        \PYG{n}{page} \PYG{o}{=} \PYG{n}{Page}\PYG{o}{.}\PYG{n}{objects}\PYG{o}{.}\PYG{n}{create}\PYG{p}{(}
            \PYG{n}{title}\PYG{o}{=}\PYG{l+s+s2}{\PYGZdq{}}\PYG{l+s+s2}{Test Page}\PYG{l+s+s2}{\PYGZdq{}}\PYG{p}{,}
            \PYG{n}{url\PYGZus{}path}\PYG{o}{=}\PYG{l+s+s2}{\PYGZdq{}}\PYG{l+s+s2}{/test}\PYG{l+s+s2}{\PYGZdq{}}\PYG{p}{,}
            \PYG{n}{language}\PYG{o}{=}\PYG{l+s+s2}{\PYGZdq{}}\PYG{l+s+s2}{EN}\PYG{l+s+s2}{\PYGZdq{}}
        \PYG{p}{)}
        \PYG{n}{page}\PYG{o}{.}\PYG{n}{publish}\PYG{p}{(}\PYG{p}{)}
        \PYG{n}{response} \PYG{o}{=} \PYG{n+nb+bp}{self}\PYG{o}{.}\PYG{n}{client}\PYG{o}{.}\PYG{n}{get}\PYG{p}{(}\PYG{l+s+s2}{\PYGZdq{}}\PYG{l+s+s2}{/test/}\PYG{l+s+s2}{\PYGZdq{}}\PYG{p}{)}
        \PYG{n+nb+bp}{self}\PYG{o}{.}\PYG{n}{assertEqual}\PYG{p}{(}\PYG{n}{response}\PYG{o}{.}\PYG{n}{status\PYGZus{}code}\PYG{p}{,} \PYG{l+m+mi}{200}\PYG{p}{)}
\end{sphinxVerbatim}


\subsection{Testergebnisse}
\label{\detokenize{sections/test_durchfuehrung:testergebnisse}}
\sphinxAtStartPar
Die Tests zeigen, dass:
\begin{itemize}
\item {} 
\sphinxAtStartPar
Alle Modelle korrekt erstellt und validiert werden

\item {} 
\sphinxAtStartPar
Die Cache\sphinxhyphen{}Implementierung wie erwartet funktioniert

\item {} 
\sphinxAtStartPar
Das Seiten\sphinxhyphen{}Rendering und URL\sphinxhyphen{}Routing korrekt arbeiten

\item {} 
\sphinxAtStartPar
Die Datenbank\sphinxhyphen{}Beziehungen zwischen den Modellen funktionieren

\item {} 
\sphinxAtStartPar
Die Sortierung von Menüpunkten korrekt implementiert ist

\end{itemize}


\subsection{Testprotokolle}
\label{\detokenize{sections/test_durchfuehrung:testprotokolle}}
\begin{sphinxVerbatim}[commandchars=\\\{\}]
============================= test session starts ==============================
platform linux \PYGZhy{}\PYGZhy{} Python 3.8.5, pytest\PYGZhy{}6.2.4
django: settings: ris\PYGZus{}dev.settings
plugins: django\PYGZhy{}4.4.0, cov\PYGZhy{}2.12.1
collected 23 items

pages\PYGZus{}app/tests/test\PYGZus{}cache.py .... [ 16\PYGZpc{}]
pages\PYGZus{}app/tests/test\PYGZus{}models.py ........... [ 60\PYGZpc{}]
pages\PYGZus{}app/tests/test\PYGZus{}views.py ...... [100\PYGZpc{}]

============================== 23 passed ==============================
\end{sphinxVerbatim}

\sphinxstepscope


\section{Models}
\label{\detokenize{sections/models:models}}\label{\detokenize{sections/models::doc}}
\sphinxAtStartPar
Diese Sektion beschreibt die Datenmodelle des RIS Backend Systems.


\subsection{TimestampedModel(base)}
\label{\detokenize{sections/models:timestampedmodel-base}}\index{module@\spxentry{module}!pages\_app.models.base@\spxentry{pages\_app.models.base}}\index{pages\_app.models.base@\spxentry{pages\_app.models.base}!module@\spxentry{module}}\phantomsection\label{\detokenize{sections/models:module-pages_app.models.base}}
\sphinxAtStartPar
Base model that adds timestamp tracking.

\sphinxAtStartPar
This module provides the TimestampedModel abstract base class that automatically
tracks creation and modification dates for all models that inherit from it.
\index{created\_at (in Modul pages\_app.models.base)@\spxentry{created\_at}\spxextra{in Modul pages\_app.models.base}}

\begin{fulllineitems}
\phantomsection\label{\detokenize{sections/models:pages_app.models.base.created_at}}
\pysigstartsignatures
\pysigline
{\sphinxcode{\sphinxupquote{pages\_app.models.base.}}\sphinxbfcode{\sphinxupquote{created\_at}}}
\pysigstopsignatures
\sphinxAtStartPar
Timestamp when record was created
\begin{quote}\begin{description}
\sphinxlineitem{Type}
\sphinxAtStartPar
DateTimeField

\end{description}\end{quote}

\end{fulllineitems}

\index{updated\_at (in Modul pages\_app.models.base)@\spxentry{updated\_at}\spxextra{in Modul pages\_app.models.base}}

\begin{fulllineitems}
\phantomsection\label{\detokenize{sections/models:pages_app.models.base.updated_at}}
\pysigstartsignatures
\pysigline
{\sphinxcode{\sphinxupquote{pages\_app.models.base.}}\sphinxbfcode{\sphinxupquote{updated\_at}}}
\pysigstopsignatures
\sphinxAtStartPar
Timestamp when record was last modified
\begin{quote}\begin{description}
\sphinxlineitem{Type}
\sphinxAtStartPar
DateTimeField

\end{description}\end{quote}

\end{fulllineitems}

\index{TimestampedModel (Klasse in pages\_app.models.base)@\spxentry{TimestampedModel}\spxextra{Klasse in pages\_app.models.base}}

\begin{fulllineitems}
\phantomsection\label{\detokenize{sections/models:pages_app.models.base.TimestampedModel}}
\pysigstartsignatures
\pysiglinewithargsret
{\sphinxbfcode{\sphinxupquote{class\DUrole{w}{ }}}\sphinxcode{\sphinxupquote{pages\_app.models.base.}}\sphinxbfcode{\sphinxupquote{TimestampedModel}}}
{\sphinxparam{\DUrole{o}{*}\DUrole{n}{args}}\sphinxparamcomma \sphinxparam{\DUrole{o}{**}\DUrole{n}{kwargs}}}
{}
\pysigstopsignatures
\sphinxAtStartPar
Bases: \sphinxcode{\sphinxupquote{Model}}

\sphinxAtStartPar
Abstract base model that automatically tracks creation and modification dates.
\index{created\_at (Attribut von pages\_app.models.base.TimestampedModel)@\spxentry{created\_at}\spxextra{Attribut von pages\_app.models.base.TimestampedModel}}

\begin{fulllineitems}
\phantomsection\label{\detokenize{sections/models:pages_app.models.base.TimestampedModel.created_at}}
\pysigstartsignatures
\pysigline
{\sphinxbfcode{\sphinxupquote{created\_at}}}
\pysigstopsignatures
\sphinxAtStartPar
Timestamp of when the record was created
\begin{quote}\begin{description}
\sphinxlineitem{Type}
\sphinxAtStartPar
DateTime

\end{description}\end{quote}

\end{fulllineitems}

\index{updated\_at (Attribut von pages\_app.models.base.TimestampedModel)@\spxentry{updated\_at}\spxextra{Attribut von pages\_app.models.base.TimestampedModel}}

\begin{fulllineitems}
\phantomsection\label{\detokenize{sections/models:pages_app.models.base.TimestampedModel.updated_at}}
\pysigstartsignatures
\pysigline
{\sphinxbfcode{\sphinxupquote{updated\_at}}}
\pysigstopsignatures
\sphinxAtStartPar
Timestamp of when the record was last modified
\begin{quote}\begin{description}
\sphinxlineitem{Type}
\sphinxAtStartPar
DateTime

\end{description}\end{quote}

\end{fulllineitems}


\begin{sphinxadmonition}{note}{Bemerkung:}
\sphinxAtStartPar
This is an abstract base class and should not be used directly.
Inherit from this class to add timestamp functionality to your models.
\end{sphinxadmonition}
\index{TimestampedModel.Meta (Klasse in pages\_app.models.base)@\spxentry{TimestampedModel.Meta}\spxextra{Klasse in pages\_app.models.base}}

\begin{fulllineitems}
\phantomsection\label{\detokenize{sections/models:pages_app.models.base.TimestampedModel.Meta}}
\pysigstartsignatures
\pysigline
{\sphinxbfcode{\sphinxupquote{class\DUrole{w}{ }}}\sphinxbfcode{\sphinxupquote{Meta}}}
\pysigstopsignatures
\sphinxAtStartPar
Bases: \sphinxhref{https://docs.python.org/3/library/functions.html\#object}{\sphinxcode{\sphinxupquote{object}}}
\index{abstract (Attribut von pages\_app.models.base.TimestampedModel.Meta)@\spxentry{abstract}\spxextra{Attribut von pages\_app.models.base.TimestampedModel.Meta}}

\begin{fulllineitems}
\phantomsection\label{\detokenize{sections/models:pages_app.models.base.TimestampedModel.Meta.abstract}}
\pysigstartsignatures
\pysigline
{\sphinxbfcode{\sphinxupquote{abstract}}\sphinxbfcode{\sphinxupquote{\DUrole{w}{ }\DUrole{p}{=}\DUrole{w}{ }False}}}
\pysigstopsignatures
\end{fulllineitems}


\end{fulllineitems}

\index{created\_at (Attribut von pages\_app.models.base.TimestampedModel)@\spxentry{created\_at}\spxextra{Attribut von pages\_app.models.base.TimestampedModel}}

\begin{fulllineitems}
\phantomsection\label{\detokenize{sections/models:id0}}
\pysigstartsignatures
\pysigline
{\sphinxbfcode{\sphinxupquote{created\_at}}}
\pysigstopsignatures
\sphinxAtStartPar
A wrapper for a deferred\sphinxhyphen{}loading field. When the value is read from this
object the first time, the query is executed.

\end{fulllineitems}

\index{get\_next\_by\_created\_at() (Methode von pages\_app.models.base.TimestampedModel)@\spxentry{get\_next\_by\_created\_at()}\spxextra{Methode von pages\_app.models.base.TimestampedModel}}

\begin{fulllineitems}
\phantomsection\label{\detokenize{sections/models:pages_app.models.base.TimestampedModel.get_next_by_created_at}}
\pysigstartsignatures
\pysiglinewithargsret
{\sphinxbfcode{\sphinxupquote{get\_next\_by\_created\_at}}}
{\sphinxparam{\DUrole{n}{*}}\sphinxparamcomma \sphinxparam{\DUrole{n}{field=\textless{}django.db.models.fields.DateTimeField: created\_at\textgreater{}}}\sphinxparamcomma \sphinxparam{\DUrole{n}{is\_next=True}}\sphinxparamcomma \sphinxparam{\DUrole{n}{**kwargs}}}
{}
\pysigstopsignatures
\end{fulllineitems}

\index{get\_next\_by\_updated\_at() (Methode von pages\_app.models.base.TimestampedModel)@\spxentry{get\_next\_by\_updated\_at()}\spxextra{Methode von pages\_app.models.base.TimestampedModel}}

\begin{fulllineitems}
\phantomsection\label{\detokenize{sections/models:pages_app.models.base.TimestampedModel.get_next_by_updated_at}}
\pysigstartsignatures
\pysiglinewithargsret
{\sphinxbfcode{\sphinxupquote{get\_next\_by\_updated\_at}}}
{\sphinxparam{\DUrole{n}{*}}\sphinxparamcomma \sphinxparam{\DUrole{n}{field=\textless{}django.db.models.fields.DateTimeField: updated\_at\textgreater{}}}\sphinxparamcomma \sphinxparam{\DUrole{n}{is\_next=True}}\sphinxparamcomma \sphinxparam{\DUrole{n}{**kwargs}}}
{}
\pysigstopsignatures
\end{fulllineitems}

\index{get\_previous\_by\_created\_at() (Methode von pages\_app.models.base.TimestampedModel)@\spxentry{get\_previous\_by\_created\_at()}\spxextra{Methode von pages\_app.models.base.TimestampedModel}}

\begin{fulllineitems}
\phantomsection\label{\detokenize{sections/models:pages_app.models.base.TimestampedModel.get_previous_by_created_at}}
\pysigstartsignatures
\pysiglinewithargsret
{\sphinxbfcode{\sphinxupquote{get\_previous\_by\_created\_at}}}
{\sphinxparam{\DUrole{n}{*}}\sphinxparamcomma \sphinxparam{\DUrole{n}{field=\textless{}django.db.models.fields.DateTimeField: created\_at\textgreater{}}}\sphinxparamcomma \sphinxparam{\DUrole{n}{is\_next=False}}\sphinxparamcomma \sphinxparam{\DUrole{n}{**kwargs}}}
{}
\pysigstopsignatures
\end{fulllineitems}

\index{get\_previous\_by\_updated\_at() (Methode von pages\_app.models.base.TimestampedModel)@\spxentry{get\_previous\_by\_updated\_at()}\spxextra{Methode von pages\_app.models.base.TimestampedModel}}

\begin{fulllineitems}
\phantomsection\label{\detokenize{sections/models:pages_app.models.base.TimestampedModel.get_previous_by_updated_at}}
\pysigstartsignatures
\pysiglinewithargsret
{\sphinxbfcode{\sphinxupquote{get\_previous\_by\_updated\_at}}}
{\sphinxparam{\DUrole{n}{*}}\sphinxparamcomma \sphinxparam{\DUrole{n}{field=\textless{}django.db.models.fields.DateTimeField: updated\_at\textgreater{}}}\sphinxparamcomma \sphinxparam{\DUrole{n}{is\_next=False}}\sphinxparamcomma \sphinxparam{\DUrole{n}{**kwargs}}}
{}
\pysigstopsignatures
\end{fulllineitems}

\index{updated\_at (Attribut von pages\_app.models.base.TimestampedModel)@\spxentry{updated\_at}\spxextra{Attribut von pages\_app.models.base.TimestampedModel}}

\begin{fulllineitems}
\phantomsection\label{\detokenize{sections/models:id1}}
\pysigstartsignatures
\pysigline
{\sphinxbfcode{\sphinxupquote{updated\_at}}}
\pysigstopsignatures
\sphinxAtStartPar
A wrapper for a deferred\sphinxhyphen{}loading field. When the value is read from this
object the first time, the query is executed.

\end{fulllineitems}


\end{fulllineitems}



\subsection{Page Model}
\label{\detokenize{sections/models:page-model}}\index{module@\spxentry{module}!pages\_app.models.Page@\spxentry{pages\_app.models.Page}}\index{pages\_app.models.Page@\spxentry{pages\_app.models.Page}!module@\spxentry{module}}\phantomsection\label{\detokenize{sections/models:module-pages_app.models.Page}}
\sphinxAtStartPar
Represents a dynamic page in the CMS.
Inherits from TimestampedModel for automatic timestamp tracking.
\index{title (in Modul pages\_app.models.Page)@\spxentry{title}\spxextra{in Modul pages\_app.models.Page}}

\begin{fulllineitems}
\phantomsection\label{\detokenize{sections/models:pages_app.models.Page.title}}
\pysigstartsignatures
\pysigline
{\sphinxcode{\sphinxupquote{pages\_app.models.Page.}}\sphinxbfcode{\sphinxupquote{title}}}
\pysigstopsignatures
\sphinxAtStartPar
Page title
\begin{quote}\begin{description}
\sphinxlineitem{Type}
\sphinxAtStartPar
\sphinxhref{https://docs.python.org/3/library/stdtypes.html\#str}{str}

\end{description}\end{quote}

\end{fulllineitems}

\index{url\_path (in Modul pages\_app.models.Page)@\spxentry{url\_path}\spxextra{in Modul pages\_app.models.Page}}

\begin{fulllineitems}
\phantomsection\label{\detokenize{sections/models:pages_app.models.Page.url_path}}
\pysigstartsignatures
\pysigline
{\sphinxcode{\sphinxupquote{pages\_app.models.Page.}}\sphinxbfcode{\sphinxupquote{url\_path}}}
\pysigstopsignatures
\sphinxAtStartPar
URL path for the page
\begin{quote}\begin{description}
\sphinxlineitem{Type}
\sphinxAtStartPar
\sphinxhref{https://docs.python.org/3/library/stdtypes.html\#str}{str}

\end{description}\end{quote}

\end{fulllineitems}

\index{name (in Modul pages\_app.models.Page)@\spxentry{name}\spxextra{in Modul pages\_app.models.Page}}

\begin{fulllineitems}
\phantomsection\label{\detokenize{sections/models:pages_app.models.Page.name}}
\pysigstartsignatures
\pysigline
{\sphinxcode{\sphinxupquote{pages\_app.models.Page.}}\sphinxbfcode{\sphinxupquote{name}}}
\pysigstopsignatures
\sphinxAtStartPar
Page identifier, max 100 chars
\begin{quote}\begin{description}
\sphinxlineitem{Type}
\sphinxAtStartPar
\sphinxhref{https://docs.python.org/3/library/stdtypes.html\#str}{str}

\end{description}\end{quote}

\end{fulllineitems}

\index{language (in Modul pages\_app.models.Page)@\spxentry{language}\spxextra{in Modul pages\_app.models.Page}}

\begin{fulllineitems}
\phantomsection\label{\detokenize{sections/models:pages_app.models.Page.language}}
\pysigstartsignatures
\pysigline
{\sphinxcode{\sphinxupquote{pages\_app.models.Page.}}\sphinxbfcode{\sphinxupquote{language}}}
\pysigstopsignatures
\sphinxAtStartPar
Language code (EN/DE)
\begin{quote}\begin{description}
\sphinxlineitem{Type}
\sphinxAtStartPar
\sphinxhref{https://docs.python.org/3/library/stdtypes.html\#str}{str}

\end{description}\end{quote}

\end{fulllineitems}

\index{meta\_description (in Modul pages\_app.models.Page)@\spxentry{meta\_description}\spxextra{in Modul pages\_app.models.Page}}

\begin{fulllineitems}
\phantomsection\label{\detokenize{sections/models:pages_app.models.Page.meta_description}}
\pysigstartsignatures
\pysigline
{\sphinxcode{\sphinxupquote{pages\_app.models.Page.}}\sphinxbfcode{\sphinxupquote{meta\_description}}}
\pysigstopsignatures
\sphinxAtStartPar
SEO meta description
\begin{quote}\begin{description}
\sphinxlineitem{Type}
\sphinxAtStartPar
\sphinxhref{https://docs.python.org/3/library/stdtypes.html\#str}{str}

\end{description}\end{quote}

\end{fulllineitems}

\index{meta\_keywords (in Modul pages\_app.models.Page)@\spxentry{meta\_keywords}\spxextra{in Modul pages\_app.models.Page}}

\begin{fulllineitems}
\phantomsection\label{\detokenize{sections/models:pages_app.models.Page.meta_keywords}}
\pysigstartsignatures
\pysigline
{\sphinxcode{\sphinxupquote{pages\_app.models.Page.}}\sphinxbfcode{\sphinxupquote{meta\_keywords}}}
\pysigstopsignatures
\sphinxAtStartPar
SEO meta keywords
\begin{quote}\begin{description}
\sphinxlineitem{Type}
\sphinxAtStartPar
\sphinxhref{https://docs.python.org/3/library/stdtypes.html\#str}{str}

\end{description}\end{quote}

\end{fulllineitems}

\index{published\_at (in Modul pages\_app.models.Page)@\spxentry{published\_at}\spxextra{in Modul pages\_app.models.Page}}

\begin{fulllineitems}
\phantomsection\label{\detokenize{sections/models:pages_app.models.Page.published_at}}
\pysigstartsignatures
\pysigline
{\sphinxcode{\sphinxupquote{pages\_app.models.Page.}}\sphinxbfcode{\sphinxupquote{published\_at}}}
\pysigstopsignatures
\sphinxAtStartPar
When the page was published
\begin{quote}\begin{description}
\sphinxlineitem{Type}
\sphinxAtStartPar
datetime

\end{description}\end{quote}

\end{fulllineitems}

\index{history (in Modul pages\_app.models.Page)@\spxentry{history}\spxextra{in Modul pages\_app.models.Page}}

\begin{fulllineitems}
\phantomsection\label{\detokenize{sections/models:pages_app.models.Page.history}}
\pysigstartsignatures
\pysigline
{\sphinxcode{\sphinxupquote{pages\_app.models.Page.}}\sphinxbfcode{\sphinxupquote{history}}}
\pysigstopsignatures
\sphinxAtStartPar
Version tracking
\begin{quote}\begin{description}
\sphinxlineitem{Type}
\sphinxAtStartPar
HistoricalRecords

\end{description}\end{quote}

\end{fulllineitems}

\subsubsection*{Example}

\begin{sphinxVerbatim}[commandchars=\\\{\}]
\PYG{g+gp}{\PYGZgt{}\PYGZgt{}\PYGZgt{} }\PYG{n}{page} \PYG{o}{=} \PYG{n}{Page}\PYG{p}{(}
\PYG{g+gp}{... }    \PYG{n}{title}\PYG{o}{=}\PYG{l+s+s2}{\PYGZdq{}}\PYG{l+s+s2}{Homepage}\PYG{l+s+s2}{\PYGZdq{}}\PYG{p}{,}
\PYG{g+gp}{... }    \PYG{n}{url\PYGZus{}path}\PYG{o}{=}\PYG{l+s+s2}{\PYGZdq{}}\PYG{l+s+s2}{/home}\PYG{l+s+s2}{\PYGZdq{}}\PYG{p}{,}
\PYG{g+gp}{... }    \PYG{n}{name}\PYG{o}{=}\PYG{l+s+s2}{\PYGZdq{}}\PYG{l+s+s2}{Homepage}\PYG{l+s+s2}{\PYGZdq{}}\PYG{p}{,}
\PYG{g+gp}{... }    \PYG{n}{language}\PYG{o}{=}\PYG{l+s+s2}{\PYGZdq{}}\PYG{l+s+s2}{EN}\PYG{l+s+s2}{\PYGZdq{}}
\PYG{g+gp}{... }\PYG{p}{)}
\PYG{g+gp}{\PYGZgt{}\PYGZgt{}\PYGZgt{} }\PYG{n}{page}\PYG{o}{.}\PYG{n}{save}\PYG{p}{(}\PYG{p}{)}
\end{sphinxVerbatim}


\subsection{Block Model}
\label{\detokenize{sections/models:block-model}}\index{module@\spxentry{module}!pages\_app.models.Block@\spxentry{pages\_app.models.Block}}\index{pages\_app.models.Block@\spxentry{pages\_app.models.Block}!module@\spxentry{module}}\phantomsection\label{\detokenize{sections/models:module-pages_app.models.Block}}
\sphinxAtStartPar
Represents a reusable content block in the CMS.

\sphinxAtStartPar
Each block can be assigned to multiple pages through PageBlock relationships.
Content is rendered using a specified template, with optional image attachment.
\index{name (in Modul pages\_app.models.Block)@\spxentry{name}\spxextra{in Modul pages\_app.models.Block}}

\begin{fulllineitems}
\phantomsection\label{\detokenize{sections/models:pages_app.models.Block.name}}
\pysigstartsignatures
\pysigline
{\sphinxcode{\sphinxupquote{pages\_app.models.Block.}}\sphinxbfcode{\sphinxupquote{name}}}
\pysigstopsignatures
\sphinxAtStartPar
Unique identifier for the block
\begin{quote}\begin{description}
\sphinxlineitem{Type}
\sphinxAtStartPar
\sphinxhref{https://docs.python.org/3/library/stdtypes.html\#str}{str}

\end{description}\end{quote}

\end{fulllineitems}

\index{template (in Modul pages\_app.models.Block)@\spxentry{template}\spxextra{in Modul pages\_app.models.Block}}

\begin{fulllineitems}
\phantomsection\label{\detokenize{sections/models:pages_app.models.Block.template}}
\pysigstartsignatures
\pysigline
{\sphinxcode{\sphinxupquote{pages\_app.models.Block.}}\sphinxbfcode{\sphinxupquote{template}}}
\pysigstopsignatures
\sphinxAtStartPar
Path to the template used for rendering
\begin{quote}\begin{description}
\sphinxlineitem{Type}
\sphinxAtStartPar
\sphinxhref{https://docs.python.org/3/library/stdtypes.html\#str}{str}

\end{description}\end{quote}

\end{fulllineitems}

\index{content (in Modul pages\_app.models.Block)@\spxentry{content}\spxextra{in Modul pages\_app.models.Block}}

\begin{fulllineitems}
\phantomsection\label{\detokenize{sections/models:pages_app.models.Block.content}}
\pysigstartsignatures
\pysigline
{\sphinxcode{\sphinxupquote{pages\_app.models.Block.}}\sphinxbfcode{\sphinxupquote{content}}}
\pysigstopsignatures
\sphinxAtStartPar
Main content of the block
\begin{quote}\begin{description}
\sphinxlineitem{Type}
\sphinxAtStartPar
text

\end{description}\end{quote}

\end{fulllineitems}

\index{sorting (in Modul pages\_app.models.Block)@\spxentry{sorting}\spxextra{in Modul pages\_app.models.Block}}

\begin{fulllineitems}
\phantomsection\label{\detokenize{sections/models:pages_app.models.Block.sorting}}
\pysigstartsignatures
\pysigline
{\sphinxcode{\sphinxupquote{pages\_app.models.Block.}}\sphinxbfcode{\sphinxupquote{sorting}}}
\pysigstopsignatures
\sphinxAtStartPar
Position in page layout
\begin{quote}\begin{description}
\sphinxlineitem{Type}
\sphinxAtStartPar
\sphinxhref{https://docs.python.org/3/library/functions.html\#int}{int}

\end{description}\end{quote}

\end{fulllineitems}

\index{image (in Modul pages\_app.models.Block)@\spxentry{image}\spxextra{in Modul pages\_app.models.Block}}

\begin{fulllineitems}
\phantomsection\label{\detokenize{sections/models:pages_app.models.Block.image}}
\pysigstartsignatures
\pysigline
{\sphinxcode{\sphinxupquote{pages\_app.models.Block.}}\sphinxbfcode{\sphinxupquote{image}}}
\pysigstopsignatures
\sphinxAtStartPar
Optional associated image
\begin{quote}\begin{description}
\sphinxlineitem{Type}
\sphinxAtStartPar
ImageField

\end{description}\end{quote}

\end{fulllineitems}

\index{history (in Modul pages\_app.models.Block)@\spxentry{history}\spxextra{in Modul pages\_app.models.Block}}

\begin{fulllineitems}
\phantomsection\label{\detokenize{sections/models:pages_app.models.Block.history}}
\pysigstartsignatures
\pysigline
{\sphinxcode{\sphinxupquote{pages\_app.models.Block.}}\sphinxbfcode{\sphinxupquote{history}}}
\pysigstopsignatures
\sphinxAtStartPar
Version tracking
\begin{quote}\begin{description}
\sphinxlineitem{Type}
\sphinxAtStartPar
HistoricalRecords

\end{description}\end{quote}

\end{fulllineitems}

\subsubsection*{Examples}

\begin{sphinxVerbatim}[commandchars=\\\{\}]
\PYG{g+gp}{\PYGZgt{}\PYGZgt{}\PYGZgt{} }\PYG{n}{block} \PYG{o}{=} \PYG{n}{Block}\PYG{o}{.}\PYG{n}{objects}\PYG{o}{.}\PYG{n}{create}\PYG{p}{(}
\PYG{g+gp}{... }    \PYG{n}{name}\PYG{o}{=}\PYG{l+s+s1}{\PYGZsq{}}\PYG{l+s+s1}{header}\PYG{l+s+s1}{\PYGZsq{}}\PYG{p}{,}
\PYG{g+gp}{... }    \PYG{n}{template}\PYG{o}{=}\PYG{l+s+s1}{\PYGZsq{}}\PYG{l+s+s1}{blocks/header.html}\PYG{l+s+s1}{\PYGZsq{}}\PYG{p}{,}
\PYG{g+gp}{... }    \PYG{n}{content}\PYG{o}{=}\PYG{l+s+s1}{\PYGZsq{}}\PYG{l+s+s1}{Welcome to our site}\PYG{l+s+s1}{\PYGZsq{}}
\PYG{g+gp}{... }\PYG{p}{)}
\end{sphinxVerbatim}


\subsection{MenuItem Model}
\label{\detokenize{sections/models:menuitem-model}}\index{module@\spxentry{module}!pages\_app.models.MenuItem@\spxentry{pages\_app.models.MenuItem}}\index{pages\_app.models.MenuItem@\spxentry{pages\_app.models.MenuItem}!module@\spxentry{module}}\phantomsection\label{\detokenize{sections/models:module-pages_app.models.MenuItem}}
\sphinxAtStartPar
Represents a navigation item in the CMS.
\index{page (in Modul pages\_app.models.MenuItem)@\spxentry{page}\spxextra{in Modul pages\_app.models.MenuItem}}

\begin{fulllineitems}
\phantomsection\label{\detokenize{sections/models:pages_app.models.MenuItem.page}}
\pysigstartsignatures
\pysigline
{\sphinxcode{\sphinxupquote{pages\_app.models.MenuItem.}}\sphinxbfcode{\sphinxupquote{page}}}
\pysigstopsignatures
\sphinxAtStartPar
Reference to the linked page
\begin{quote}\begin{description}
\sphinxlineitem{Type}
\sphinxAtStartPar
ForeignKey

\end{description}\end{quote}

\end{fulllineitems}

\index{name (in Modul pages\_app.models.MenuItem)@\spxentry{name}\spxextra{in Modul pages\_app.models.MenuItem}}

\begin{fulllineitems}
\phantomsection\label{\detokenize{sections/models:pages_app.models.MenuItem.name}}
\pysigstartsignatures
\pysigline
{\sphinxcode{\sphinxupquote{pages\_app.models.MenuItem.}}\sphinxbfcode{\sphinxupquote{name}}}
\pysigstopsignatures
\sphinxAtStartPar
Name of the menu item
\begin{quote}\begin{description}
\sphinxlineitem{Type}
\sphinxAtStartPar
\sphinxhref{https://docs.python.org/3/library/stdtypes.html\#str}{str}

\end{description}\end{quote}

\end{fulllineitems}

\index{template (in Modul pages\_app.models.MenuItem)@\spxentry{template}\spxextra{in Modul pages\_app.models.MenuItem}}

\begin{fulllineitems}
\phantomsection\label{\detokenize{sections/models:pages_app.models.MenuItem.template}}
\pysigstartsignatures
\pysigline
{\sphinxcode{\sphinxupquote{pages\_app.models.MenuItem.}}\sphinxbfcode{\sphinxupquote{template}}}
\pysigstopsignatures
\sphinxAtStartPar
Template content for the menu item
\begin{quote}\begin{description}
\sphinxlineitem{Type}
\sphinxAtStartPar
\sphinxhref{https://docs.python.org/3/library/stdtypes.html\#str}{str}

\end{description}\end{quote}

\end{fulllineitems}

\index{sorting (in Modul pages\_app.models.MenuItem)@\spxentry{sorting}\spxextra{in Modul pages\_app.models.MenuItem}}

\begin{fulllineitems}
\phantomsection\label{\detokenize{sections/models:pages_app.models.MenuItem.sorting}}
\pysigstartsignatures
\pysigline
{\sphinxcode{\sphinxupquote{pages\_app.models.MenuItem.}}\sphinxbfcode{\sphinxupquote{sorting}}}
\pysigstopsignatures
\sphinxAtStartPar
Sorting order
\begin{quote}\begin{description}
\sphinxlineitem{Type}
\sphinxAtStartPar
\sphinxhref{https://docs.python.org/3/library/functions.html\#int}{int}

\end{description}\end{quote}

\end{fulllineitems}

\index{history (in Modul pages\_app.models.MenuItem)@\spxentry{history}\spxextra{in Modul pages\_app.models.MenuItem}}

\begin{fulllineitems}
\phantomsection\label{\detokenize{sections/models:pages_app.models.MenuItem.history}}
\pysigstartsignatures
\pysigline
{\sphinxcode{\sphinxupquote{pages\_app.models.MenuItem.}}\sphinxbfcode{\sphinxupquote{history}}}
\pysigstopsignatures
\sphinxAtStartPar
Tracks changes to the menu item
\begin{quote}\begin{description}
\sphinxlineitem{Type}
\sphinxAtStartPar
HistoricalRecords

\end{description}\end{quote}

\end{fulllineitems}

\subsubsection*{Examples}

\begin{sphinxVerbatim}[commandchars=\\\{\}]
\PYG{g+gp}{\PYGZgt{}\PYGZgt{}\PYGZgt{} }\PYG{n}{page} \PYG{o}{=} \PYG{n}{Page}\PYG{o}{.}\PYG{n}{objects}\PYG{o}{.}\PYG{n}{get}\PYG{p}{(}\PYG{n}{slug}\PYG{o}{=}\PYG{l+s+s2}{\PYGZdq{}}\PYG{l+s+s2}{home}\PYG{l+s+s2}{\PYGZdq{}}\PYG{p}{)}
\PYG{g+gp}{\PYGZgt{}\PYGZgt{}\PYGZgt{} }\PYG{n}{menu\PYGZus{}item} \PYG{o}{=} \PYG{n}{MenuItem}\PYG{o}{.}\PYG{n}{objects}\PYG{o}{.}\PYG{n}{create}\PYG{p}{(}
\PYG{g+gp}{... }    \PYG{n}{page}\PYG{o}{=}\PYG{n}{page}\PYG{p}{,}
\PYG{g+gp}{... }    \PYG{n}{name}\PYG{o}{=}\PYG{l+s+s2}{\PYGZdq{}}\PYG{l+s+s2}{Home}\PYG{l+s+s2}{\PYGZdq{}}\PYG{p}{,}
\PYG{g+gp}{... }    \PYG{n}{template}\PYG{o}{=}\PYG{l+s+s2}{\PYGZdq{}}\PYG{l+s+s2}{\PYGZlt{}a href=}\PYG{l+s+s2}{\PYGZsq{}}\PYG{l+s+s2}{/home}\PYG{l+s+s2}{\PYGZsq{}}\PYG{l+s+s2}{\PYGZgt{}Home\PYGZlt{}/a\PYGZgt{}}\PYG{l+s+s2}{\PYGZdq{}}\PYG{p}{,}
\PYG{g+gp}{... }    \PYG{n}{sorting}\PYG{o}{=}\PYG{l+m+mi}{1}
\PYG{g+gp}{... }\PYG{p}{)}
\end{sphinxVerbatim}


\subsection{PageBlock Model}
\label{\detokenize{sections/models:pageblock-model}}\index{module@\spxentry{module}!pages\_app.models.PageBlock@\spxentry{pages\_app.models.PageBlock}}\index{pages\_app.models.PageBlock@\spxentry{pages\_app.models.PageBlock}!module@\spxentry{module}}\phantomsection\label{\detokenize{sections/models:module-pages_app.models.PageBlock}}
\sphinxAtStartPar
Associates blocks with pages and manages their positioning.
\index{page (in Modul pages\_app.models.PageBlock)@\spxentry{page}\spxextra{in Modul pages\_app.models.PageBlock}}

\begin{fulllineitems}
\phantomsection\label{\detokenize{sections/models:pages_app.models.PageBlock.page}}
\pysigstartsignatures
\pysigline
{\sphinxcode{\sphinxupquote{pages\_app.models.PageBlock.}}\sphinxbfcode{\sphinxupquote{page}}}
\pysigstopsignatures
\sphinxAtStartPar
Reference to the parent page
\begin{quote}\begin{description}
\sphinxlineitem{Type}
\sphinxAtStartPar
ForeignKey

\end{description}\end{quote}

\end{fulllineitems}

\index{block (in Modul pages\_app.models.PageBlock)@\spxentry{block}\spxextra{in Modul pages\_app.models.PageBlock}}

\begin{fulllineitems}
\phantomsection\label{\detokenize{sections/models:pages_app.models.PageBlock.block}}
\pysigstartsignatures
\pysigline
{\sphinxcode{\sphinxupquote{pages\_app.models.PageBlock.}}\sphinxbfcode{\sphinxupquote{block}}}
\pysigstopsignatures
\sphinxAtStartPar
Reference to the content block
\begin{quote}\begin{description}
\sphinxlineitem{Type}
\sphinxAtStartPar
ForeignKey

\end{description}\end{quote}

\end{fulllineitems}

\index{position (in Modul pages\_app.models.PageBlock)@\spxentry{position}\spxextra{in Modul pages\_app.models.PageBlock}}

\begin{fulllineitems}
\phantomsection\label{\detokenize{sections/models:pages_app.models.PageBlock.position}}
\pysigstartsignatures
\pysigline
{\sphinxcode{\sphinxupquote{pages\_app.models.PageBlock.}}\sphinxbfcode{\sphinxupquote{position}}}
\pysigstopsignatures
\sphinxAtStartPar
Order position within the page
\begin{quote}\begin{description}
\sphinxlineitem{Type}
\sphinxAtStartPar
IntegerField

\end{description}\end{quote}

\end{fulllineitems}

\index{history (in Modul pages\_app.models.PageBlock)@\spxentry{history}\spxextra{in Modul pages\_app.models.PageBlock}}

\begin{fulllineitems}
\phantomsection\label{\detokenize{sections/models:pages_app.models.PageBlock.history}}
\pysigstartsignatures
\pysigline
{\sphinxcode{\sphinxupquote{pages\_app.models.PageBlock.}}\sphinxbfcode{\sphinxupquote{history}}}
\pysigstopsignatures
\sphinxAtStartPar
Tracks changes to assignments
\begin{quote}\begin{description}
\sphinxlineitem{Type}
\sphinxAtStartPar
HistoricalRecords

\end{description}\end{quote}

\end{fulllineitems}

\subsubsection*{Examples}

\begin{sphinxVerbatim}[commandchars=\\\{\}]
\PYG{g+gp}{\PYGZgt{}\PYGZgt{}\PYGZgt{} }\PYG{n}{header} \PYG{o}{=} \PYG{n}{Block}\PYG{o}{.}\PYG{n}{objects}\PYG{o}{.}\PYG{n}{get}\PYG{p}{(}\PYG{n}{name}\PYG{o}{=}\PYG{l+s+s2}{\PYGZdq{}}\PYG{l+s+s2}{Header}\PYG{l+s+s2}{\PYGZdq{}}\PYG{p}{)}
\PYG{g+gp}{\PYGZgt{}\PYGZgt{}\PYGZgt{} }\PYG{n}{home\PYGZus{}page} \PYG{o}{=} \PYG{n}{Page}\PYG{o}{.}\PYG{n}{objects}\PYG{o}{.}\PYG{n}{get}\PYG{p}{(}\PYG{n}{slug}\PYG{o}{=}\PYG{l+s+s2}{\PYGZdq{}}\PYG{l+s+s2}{home}\PYG{l+s+s2}{\PYGZdq{}}\PYG{p}{)}
\PYG{g+gp}{\PYGZgt{}\PYGZgt{}\PYGZgt{} }\PYG{n}{page\PYGZus{}block} \PYG{o}{=} \PYG{n}{PageBlock}\PYG{o}{.}\PYG{n}{objects}\PYG{o}{.}\PYG{n}{create}\PYG{p}{(}
\PYG{g+gp}{... }    \PYG{n}{page}\PYG{o}{=}\PYG{n}{home\PYGZus{}page}\PYG{p}{,}
\PYG{g+gp}{... }    \PYG{n}{block}\PYG{o}{=}\PYG{n}{header}\PYG{p}{,}
\PYG{g+gp}{... }    \PYG{n}{position}\PYG{o}{=}\PYG{l+m+mi}{1}
\PYG{g+gp}{... }\PYG{p}{)}
\end{sphinxVerbatim}


\subsection{MTV\sphinxhyphen{}Architektur}
\label{\detokenize{sections/models:mtv-architektur}}
\sphinxAtStartPar
Das Backend verwendet die MTV\sphinxhyphen{}Architektur (Model\sphinxhyphen{}Template\sphinxhyphen{}View). Für eine detaillierte Darstellung der Architektur, siehe {\hyperref[\detokenize{sections/diagramme:mtv-architecture}]{\sphinxcrossref{\DUrole{std}{\DUrole{std-ref}{MTV\sphinxhyphen{}Architektur}}}}}.

\sphinxstepscope


\section{Glosar}
\label{\detokenize{sections/glosar:glosar}}\label{\detokenize{sections/glosar::doc}}
\sphinxAtStartPar
Ein Glossar mit Definitionen von technischen Begriffen und Konzepten.


\subsection{Begriffe}
\label{\detokenize{sections/glosar:begriffe}}
\sphinxAtStartPar
\sphinxstylestrong{Backend}
Der Teil einer Anwendung, der für Datenverarbeitung und Logik zuständig ist.

\sphinxAtStartPar
\sphinxstylestrong{Frontend}
Der Teil einer Anwendung, der für die Benutzerschnittstelle verantwortlich ist.

\sphinxAtStartPar
\sphinxstylestrong{ORM}
Object\sphinxhyphen{}Relational Mapping, eine Technik, um Datenbankabfragen in Code zu integrieren.

\sphinxAtStartPar
\sphinxstylestrong{Caching}
Zwischenspeicherung von Daten, um Zugriffe zu beschleunigen.

\sphinxAtStartPar
\sphinxstylestrong{Middleware}
Software, die zwischen Betriebssystem und Anwendungen vermittelt.

\sphinxAtStartPar
\sphinxstylestrong{API}
Schnittstelle zur Kommunikation zwischen Softwarekomponenten.

\sphinxAtStartPar
\sphinxstylestrong{Django Signals}
Mechanismus zum automatischen Auslösen von Funktionen bei bestimmten Ereignissen.

\sphinxAtStartPar
\sphinxstylestrong{Redis}
In\sphinxhyphen{}Memory\sphinxhyphen{}Datenspeicher zur Beschleunigung von Datenzugriffen.

\sphinxAtStartPar
\sphinxstylestrong{PostgreSQL}
Leistungsstarkes, objektrelationales Datenbankmanagementsystem.

\sphinxAtStartPar
\sphinxstylestrong{SQLite}
Leichtgewichtiges, serverloses Datenbankmanagementsystem.

\sphinxAtStartPar
\sphinxstylestrong{Serverseitiges Rendern}
Erzeugung von HTML durch den Server, bevor es an den Client gesendet wird.

\sphinxAtStartPar
\sphinxstylestrong{Template\sphinxhyphen{}System}
System zur Trennung von Logik und Darstellung in Webanwendungen.

\sphinxAtStartPar
\sphinxstylestrong{Mehrsprachigkeit}
Fähigkeit einer Anwendung, Inhalte in verschiedenen Sprachen bereitzustellen.

\sphinxAtStartPar
\sphinxstylestrong{Automatische Migrationen}
Automatisierte Anpassung der Datenbankstruktur an neue Anforderungen.

\sphinxAtStartPar
\sphinxstylestrong{Nutzwertanalyse}
Bewertungsmethode zur Auswahl von Alternativen anhand verschiedener Kriterien.

\sphinxAtStartPar
\sphinxstylestrong{Projektumfang}
Festgelegte Grenzen und Ziele eines Entwicklungsprojekts.

\sphinxAtStartPar
\sphinxstylestrong{Entwicklungsphase}
Abschnitt im Projekt, in dem Funktionen implementiert und getestet werden.

\sphinxAtStartPar
\sphinxstylestrong{Release\sphinxhyphen{}Management}
Planung und Steuerung von Softwareveröffentlichungen.

\sphinxAtStartPar
\sphinxstylestrong{Vorgehensmodell}
Strukturierter Ansatz zur Durchführung von Projekten.

\sphinxAtStartPar
\sphinxstylestrong{Wasserfallmodell}
Lineares Vorgehensmodell mit klaren Phasen und Übergängen.

\sphinxAtStartPar
\sphinxstylestrong{MTV\sphinxhyphen{}Architektur}
Django\sphinxhyphen{}spezifische Variante des MVC\sphinxhyphen{}Patterns mit Model, Template und View.

\sphinxAtStartPar
\sphinxstylestrong{Middleware\sphinxhyphen{}Cache}
Zwischenschicht für temporäre Datenspeicherung auf Anwendungsebene.

\sphinxAtStartPar
\sphinxstylestrong{Template\sphinxhyphen{}Engine}
System zur dynamischen Generierung von HTML aus wiederverwendbaren Komponenten.

\sphinxAtStartPar
\sphinxstylestrong{Migration}
Versionierte Änderungen am Datenbankschema.

\sphinxAtStartPar
\sphinxstylestrong{Model\sphinxhyphen{}Manager}
Django\sphinxhyphen{}Klasse für datenbankbezogene Operationen auf Modellebene.

\sphinxAtStartPar
\sphinxstylestrong{Signalsystem}
Event\sphinxhyphen{}basierte Kommunikation zwischen Django\sphinxhyphen{}Komponenten.

\sphinxAtStartPar
\sphinxstylestrong{QuerySet}
Lazy\sphinxhyphen{}Loading Datenbank\sphinxhyphen{}Abfragen in Django.

\sphinxAtStartPar
\sphinxstylestrong{Templatetags}
Erweiterbare Template\sphinxhyphen{}Funktionen für komplexe Darstellungslogik.

\sphinxAtStartPar
\sphinxstylestrong{Context\sphinxhyphen{}Processor}
Globale Variablen für Template\sphinxhyphen{}Rendering.

\sphinxAtStartPar
\sphinxstylestrong{Datenmigration}
Prozess zum Übertragen von Daten zwischen Datenbankschemas.

\sphinxAtStartPar
\sphinxstylestrong{Codebase}
Gesamtheit des Quellcodes einer Anwendung.

\sphinxAtStartPar
\sphinxstylestrong{Deploymentprozess}
Standardisierter Ablauf zur Produktivsetzung.

\sphinxAtStartPar
\sphinxstylestrong{Entwicklungsumgebung}
Isolierte Systemkonfiguration für die Entwicklung.

\sphinxAtStartPar
\sphinxstylestrong{Produktivumgebung}
Live\sphinxhyphen{}System mit Endbenutzer\sphinxhyphen{}Zugriff.

\sphinxAtStartPar
\sphinxstylestrong{Cachestrategie}
Konzept zur optimalen Nutzung verschiedener Cache\sphinxhyphen{}Ebenen.

\sphinxAtStartPar
\sphinxstylestrong{Continuous Integration}
Automatisierte Build\sphinxhyphen{} und Testprozesse.

\sphinxAtStartPar
\sphinxstylestrong{Versionskontrolle}
System zur Verwaltung von Codeänderungen.

\sphinxAtStartPar
\sphinxstylestrong{Content\sphinxhyphen{}Management}
Strukturierte Verwaltung von Website\sphinxhyphen{}Inhalten.

\sphinxAtStartPar
\sphinxstylestrong{Backend\sphinxhyphen{}Architektur}
Serverseitige Systemstruktur einer Webanwendung.


\subsection{Abkürzungen}
\label{\detokenize{sections/glosar:abkurzungen}}
\sphinxAtStartPar
\sphinxstylestrong{ACL}
Access Control List

\sphinxAtStartPar
\sphinxstylestrong{API}
Application Programming Interface

\sphinxAtStartPar
\sphinxstylestrong{CDN}
Content Delivery Network

\sphinxAtStartPar
\sphinxstylestrong{CI}
Continuous Integration

\sphinxAtStartPar
\sphinxstylestrong{CI/CD}
Continuous Integration/Continuous Deployment

\sphinxAtStartPar
\sphinxstylestrong{CMS}
Content Management System

\sphinxAtStartPar
\sphinxstylestrong{CORS}
Cross\sphinxhyphen{}Origin Resource Sharing

\sphinxAtStartPar
\sphinxstylestrong{CPU}
Central Processing Unit

\sphinxAtStartPar
\sphinxstylestrong{CSRF}
Cross\sphinxhyphen{}Site Request Forgery

\sphinxAtStartPar
\sphinxstylestrong{DB}
Datenbank

\sphinxAtStartPar
\sphinxstylestrong{DTL}
Django Template Language

\sphinxAtStartPar
\sphinxstylestrong{ERD}
Entity\sphinxhyphen{}Relationship\sphinxhyphen{}Diagramm

\sphinxAtStartPar
\sphinxstylestrong{GPU}
Graphics Processing Unit

\sphinxAtStartPar
\sphinxstylestrong{HTTP}
Hypertext Transfer Protocol

\sphinxAtStartPar
\sphinxstylestrong{I18N}
Internationalization

\sphinxAtStartPar
\sphinxstylestrong{IDE}
Integrated Development Environment

\sphinxAtStartPar
\sphinxstylestrong{JWT}
JSON Web Token

\sphinxAtStartPar
\sphinxstylestrong{L10N}
Localization

\sphinxAtStartPar
\sphinxstylestrong{MVC}
Model\sphinxhyphen{}View\sphinxhyphen{}Controller

\sphinxAtStartPar
\sphinxstylestrong{MTV}
Model\sphinxhyphen{}Template\sphinxhyphen{}View

\sphinxAtStartPar
\sphinxstylestrong{ORM}
Object\sphinxhyphen{}Relational Mapping

\sphinxAtStartPar
\sphinxstylestrong{RAM}
Random Access Memory

\sphinxAtStartPar
\sphinxstylestrong{RDBMS}
Relationales Datenbank\sphinxhyphen{}Management\sphinxhyphen{}System

\sphinxAtStartPar
\sphinxstylestrong{REST}
Representational State Transfer

\sphinxAtStartPar
\sphinxstylestrong{SEO}
Search Engine Optimization

\sphinxAtStartPar
\sphinxstylestrong{SLA}
Service Level Agreement

\sphinxAtStartPar
\sphinxstylestrong{SQLite}
Leichtgewichtiges, serverloses Datenbankmanagementsystem

\sphinxAtStartPar
\sphinxstylestrong{SSH}
Secure Shell

\sphinxAtStartPar
\sphinxstylestrong{SSR}
Server\sphinxhyphen{}Side Rendering

\sphinxAtStartPar
\sphinxstylestrong{TLS}
Transport Layer Security

\sphinxAtStartPar
\sphinxstylestrong{UML}
Unified Modeling Language

\sphinxAtStartPar
\sphinxstylestrong{UI}
User Interface

\sphinxAtStartPar
\sphinxstylestrong{UX}
User Experience

\sphinxAtStartPar
\sphinxstylestrong{VCS}
Version Control System

\sphinxAtStartPar
\sphinxstylestrong{WSGI}
Web Server Gateway Interface

\sphinxstepscope


\section{Diagramme}
\label{\detokenize{sections/diagramme:diagramme}}\label{\detokenize{sections/diagramme::doc}}
\sphinxAtStartPar
Diese Sektion enthält die wichtigsten Diagramme für das Projekt.


\subsection{Use Case Diagramm}
\label{\detokenize{sections/diagramme:use-case-diagramm}}\label{\detokenize{sections/diagramme:use-case-diagram}}
\noindent\sphinxincludegraphics{{plantuml-e9c24912352634617b3287bd8a977e11f8d60ab9}.png}


\subsection{ERD\sphinxhyphen{}Diagramm}
\label{\detokenize{sections/diagramme:erd-diagramm}}\label{\detokenize{sections/diagramme:erd-diagram}}
\noindent\sphinxincludegraphics{{plantuml-93f76ee9d471be835e35ae83290aa5dc6461315a}.png}


\subsection{Signalfluss\sphinxhyphen{}Diagramm}
\label{\detokenize{sections/diagramme:signalfluss-diagramm}}\label{\detokenize{sections/diagramme:signal-flow-diagram}}
\noindent\sphinxincludegraphics{{plantuml-5825e31db28e0f58d83cf25a247933a3c1111820}.png}


\subsection{Seiten\sphinxhyphen{}Rendering\sphinxhyphen{}Diagramm}
\label{\detokenize{sections/diagramme:seiten-rendering-diagramm}}\label{\detokenize{sections/diagramme:page-rendering-diagram}}
\noindent\sphinxincludegraphics{{plantuml-7986957a0782e5af722078ccb3c570ce31489846}.png}


\subsection{Navigations\sphinxhyphen{}Diagramm}
\label{\detokenize{sections/diagramme:navigations-diagramm}}\label{\detokenize{sections/diagramme:navigation-diagram}}
\noindent\sphinxincludegraphics{{plantuml-780530252f5e56b4d764ad2b6aeca8667c0fff39}.png}


\subsection{MTV\sphinxhyphen{}Architektur}
\label{\detokenize{sections/diagramme:mtv-architektur}}\label{\detokenize{sections/diagramme:mtv-architecture}}
\noindent\sphinxincludegraphics{{plantuml-262c6afa9ad198ef9e4d3295fef58fecb4b8efef}.png}

\sphinxstepscope


\section{Bilder}
\label{\detokenize{sections/bilder:bilder}}\label{\detokenize{sections/bilder::doc}}
\sphinxAtStartPar
In dieser Sektion können zusätzliche Bilder eingefügt werden, die das Projekt illustrieren.

\sphinxAtStartPar
Beispiele für mögliche Bilder:
\sphinxhyphen{} Screenshots von Benutzeroberflächen
\sphinxhyphen{} Diagramme, die in anderen Sektionen nicht abgedeckt sind
\sphinxhyphen{} Architekturkonzepte oder Implementierungsdetails

\sphinxstepscope


\section{Detallierter Zeitplanung}
\label{\detokenize{sections/tables:detallierter-zeitplanung}}\label{\detokenize{sections/tables:detalierter-zeitplanung}}\label{\detokenize{sections/tables::doc}}
\sphinxAtStartPar
Hier ist die detaillierte Zeitplanung mit den Aufgaben und ihrer Dauer:


\begin{savenotes}\sphinxattablestart
\sphinxthistablewithglobalstyle
\centering
\begin{tabulary}{\linewidth}[t]{TTT}
\sphinxtoprule
\sphinxstyletheadfamily 
\sphinxAtStartPar
\sphinxstylestrong{Phase/Aufgabe}
&\sphinxstyletheadfamily 
\sphinxAtStartPar
\sphinxstylestrong{Geplant (Std)}
&\sphinxstyletheadfamily 
\sphinxAtStartPar
\sphinxstylestrong{Tatsächlich}
\\
\sphinxmidrule
\sphinxtableatstartofbodyhook
\sphinxAtStartPar
\sphinxstylestrong{Projektplanung/Analyse}
&
\sphinxAtStartPar
8
&
\sphinxAtStartPar
8
\\
\sphinxhline
\sphinxAtStartPar
Ist\sphinxhyphen{}Analyse
&
\sphinxAtStartPar
2
&
\sphinxAtStartPar
2
\\
\sphinxhline
\sphinxAtStartPar
Anforderungsanalyse
&
\sphinxAtStartPar
2
&
\sphinxAtStartPar
2
\\
\sphinxhline
\sphinxAtStartPar
Framework\sphinxhyphen{}Vergleich
&
\sphinxAtStartPar
2
&
\sphinxAtStartPar
2
\\
\sphinxhline
\sphinxAtStartPar
Ressourcenplanung
&
\sphinxAtStartPar
2
&
\sphinxAtStartPar
2
\\
\sphinxhline
\sphinxAtStartPar
\sphinxstylestrong{Entwurf}
&
\sphinxAtStartPar
16
&
\sphinxAtStartPar
17
\\
\sphinxhline
\sphinxAtStartPar
Datenbankmodell
&
\sphinxAtStartPar
4
&
\sphinxAtStartPar
4
\\
\sphinxhline
\sphinxAtStartPar
MTV\sphinxhyphen{}Architektur
&
\sphinxAtStartPar
4
&
\sphinxAtStartPar
4
\\
\sphinxhline
\sphinxAtStartPar
Cache\sphinxhyphen{}Strategie
&
\sphinxAtStartPar
4
&
\sphinxAtStartPar
4
\\
\sphinxhline
\sphinxAtStartPar
API\sphinxhyphen{}Design
&
\sphinxAtStartPar
4
&
\sphinxAtStartPar
5
\\
\sphinxhline
\sphinxAtStartPar
\sphinxstylestrong{Implementierung}
&
\sphinxAtStartPar
36
&
\sphinxAtStartPar
35
\\
\sphinxhline
\sphinxAtStartPar
Grundstruktur
&
\sphinxAtStartPar
6
&
\sphinxAtStartPar
6
\\
\sphinxhline
\sphinxAtStartPar
Datenbank\sphinxhyphen{}Setup
&
\sphinxAtStartPar
6
&
\sphinxAtStartPar
6
\\
\sphinxhline
\sphinxAtStartPar
Models \& Views
&
\sphinxAtStartPar
8
&
\sphinxAtStartPar
7
\\
\sphinxhline
\sphinxAtStartPar
Cache\sphinxhyphen{}Implementierung
&
\sphinxAtStartPar
8
&
\sphinxAtStartPar
8
\\
\sphinxhline
\sphinxAtStartPar
Signals \& Email
&
\sphinxAtStartPar
4
&
\sphinxAtStartPar
4
\\
\sphinxhline
\sphinxAtStartPar
Admin\sphinxhyphen{}Interface
&
\sphinxAtStartPar
4
&
\sphinxAtStartPar
4
\\
\sphinxhline
\sphinxAtStartPar
\sphinxstylestrong{Test/Durchführung}
&
\sphinxAtStartPar
8
&
\sphinxAtStartPar
8
\\
\sphinxhline
\sphinxAtStartPar
Unit Tests
&
\sphinxAtStartPar
3
&
\sphinxAtStartPar
3
\\
\sphinxhline
\sphinxAtStartPar
Integrationstests
&
\sphinxAtStartPar
3
&
\sphinxAtStartPar
3
\\
\sphinxhline
\sphinxAtStartPar
Performance Tests
&
\sphinxAtStartPar
2
&
\sphinxAtStartPar
2
\\
\sphinxhline
\sphinxAtStartPar
\sphinxstylestrong{Dokumentation}
&
\sphinxAtStartPar
8
&
\sphinxAtStartPar
8
\\
\sphinxhline
\sphinxAtStartPar
Technische Doku
&
\sphinxAtStartPar
4
&
\sphinxAtStartPar
4
\\
\sphinxhline
\sphinxAtStartPar
Code\sphinxhyphen{}Dokumentation
&
\sphinxAtStartPar
2
&
\sphinxAtStartPar
2
\\
\sphinxhline
\sphinxAtStartPar
Benutzerhandbuch
&
\sphinxAtStartPar
2
&
\sphinxAtStartPar
2
\\
\sphinxhline
\sphinxAtStartPar
\sphinxstylestrong{Abnahme}
&
\sphinxAtStartPar
4
&
\sphinxAtStartPar
4
\\
\sphinxhline
\sphinxAtStartPar
Präsentation
&
\sphinxAtStartPar
2
&
\sphinxAtStartPar
2
\\
\sphinxhline
\sphinxAtStartPar
Feedback \& Anpassungen
&
\sphinxAtStartPar
2
&
\sphinxAtStartPar
2
\\
\sphinxhline
\sphinxAtStartPar
\sphinxstylestrong{Gesamtaufwand}
&
\sphinxAtStartPar
80
&
\sphinxAtStartPar
80
\\
\sphinxbottomrule
\end{tabulary}
\sphinxtableafterendhook\par
\sphinxattableend\end{savenotes}

\sphinxstepscope


\section{Fazit}
\label{\detokenize{sections/fazit:fazit}}\label{\detokenize{sections/fazit::doc}}

\subsection{Soll\sphinxhyphen{}/Ist\sphinxhyphen{}Vergleich}
\label{\detokenize{sections/fazit:soll-ist-vergleich}}
\sphinxAtStartPar
Das Projektziel, ein maßgeschneidertes Backend für RIS Web\sphinxhyphen{} \& Software\sphinxhyphen{}Development GmbH \& Co. KG zu entwickeln, wurde weitgehend erreicht. Die wichtigsten Anforderungen wurden erfüllt:
\begin{itemize}
\item {} 
\sphinxAtStartPar
\sphinxstylestrong{Volle Kontrolle}: Alle Backend\sphinxhyphen{}Funktionen wurden feinjustiert und unabhängig von allgemeinen Lösungen wie WordPress implementiert.

\item {} 
\sphinxAtStartPar
\sphinxstylestrong{Verschlankung}: Das Backend wurde gezielt auf die spezifischen Bedürfnisse von RIS zugeschnitten, um unnötige Funktionen zu vermeiden und die Bedienbarkeit zu vereinfachen.

\item {} 
\sphinxAtStartPar
\sphinxstylestrong{Effiziente Inhaltspflege}: Die Verwaltung von Inhalten erfolgt ohne die Einschränkungen und Komplexität eines WordPress\sphinxhyphen{}Templates, wodurch ein direkter und effizienter Arbeitsprozess ermöglicht wird.

\item {} 
\sphinxAtStartPar
\sphinxstylestrong{Performance}: Die Website wird serverseitig gerendert und die Seiteninhalte werden vollständig im Cache gespeichert, um optimale Ladegeschwindigkeiten zu erreichen.

\end{itemize}

\sphinxAtStartPar
\sphinxstylestrong{Software \& Technologien:}
\begin{itemize}
\item {} 
\sphinxAtStartPar
\sphinxstylestrong{Django\sphinxhyphen{}Ökosystem:}
\begin{itemize}
\item {} 
\sphinxAtStartPar
Django Framework (Web\sphinxhyphen{}Framework)

\item {} 
\sphinxAtStartPar
django\sphinxhyphen{}simple\sphinxhyphen{}history (Änderungsverfolgung für Models)

\item {} 
\sphinxAtStartPar
django\sphinxhyphen{}redis (Cache\sphinxhyphen{}Backend\sphinxhyphen{}Integration)

\item {} 
\sphinxAtStartPar
django\sphinxhyphen{}environ (Umgebungsvariablen\sphinxhyphen{}Management)

\item {} 
\sphinxAtStartPar
Django Debug Toolbar (Entwicklungs\sphinxhyphen{}Debugging)

\end{itemize}

\item {} 
\sphinxAtStartPar
\sphinxstylestrong{Datenbank \& Caching:}
\begin{itemize}
\item {} 
\sphinxAtStartPar
SQLite (Leichtgewichtige Entwicklungsdatenbank)

\item {} 
\sphinxAtStartPar
PostgreSQL (Robuste Produktionsdatenbank)

\item {} 
\sphinxAtStartPar
Redis (In\sphinxhyphen{}Memory Caching)

\end{itemize}

\item {} 
\sphinxAtStartPar
\sphinxstylestrong{Server \& Deployment:}
\begin{itemize}
\item {} 
\sphinxAtStartPar
Gunicorn (Python WSGI HTTP Server)

\item {} 
\sphinxAtStartPar
Nginx (Hochleistungs\sphinxhyphen{}Webserver)

\item {} 
\sphinxAtStartPar
WhiteNoise (Statische Dateien\sphinxhyphen{}Verwaltung)

\item {} 
\sphinxAtStartPar
systemd (Service\sphinxhyphen{}Management)

\end{itemize}

\item {} 
\sphinxAtStartPar
\sphinxstylestrong{Testing \& QA:}
\begin{itemize}
\item {} 
\sphinxAtStartPar
PyTest \& PyTest\sphinxhyphen{}Django (Test\sphinxhyphen{}Framework)

\item {} 
\sphinxAtStartPar
FactoryBoy (Testdaten\sphinxhyphen{}Generator)

\item {} 
\sphinxAtStartPar
Coverage.py (Code\sphinxhyphen{}Abdeckungsanalyse)

\end{itemize}

\item {} 
\sphinxAtStartPar
\sphinxstylestrong{Dokumentation:}
\begin{itemize}
\item {} 
\sphinxAtStartPar
Sphinx (Dokumentations\sphinxhyphen{}Generator)

\item {} 
\sphinxAtStartPar
sphinx\sphinxhyphen{}rtd\sphinxhyphen{}theme (ReadTheDocs Theme)

\item {} 
\sphinxAtStartPar
sphinxcontrib\sphinxhyphen{}plantuml (UML\sphinxhyphen{}Diagramme)

\end{itemize}

\item {} 
\sphinxAtStartPar
\sphinxstylestrong{Entwicklungstools:}
\begin{itemize}
\item {} 
\sphinxAtStartPar
Python 3.8+ (Programmiersprache)

\item {} 
\sphinxAtStartPar
PyCharm IDE (Entwicklungsumgebung)

\item {} 
\sphinxAtStartPar
Git \& GitLab (Versionskontrolle)

\item {} 
\sphinxAtStartPar
make (Build\sphinxhyphen{}Automatisierung)

\end{itemize}

\item {} 
\sphinxAtStartPar
\sphinxstylestrong{Media \& Assets:}
\begin{itemize}
\item {} 
\sphinxAtStartPar
Pillow (Python Imaging Library)

\end{itemize}

\end{itemize}


\subsection{Projektziel erreicht}
\label{\detokenize{sections/fazit:projektziel-erreicht}}
\sphinxAtStartPar
Das Projektziel wurde erreicht. Alle geplanten Funktionen und Anforderungen wurden erfolgreich implementiert und getestet. Es gab keine wesentlichen Abweichungen vom ursprünglichen Plan.


\subsection{Zeitunterschied}
\label{\detokenize{sections/fazit:zeitunterschied}}
\sphinxAtStartPar
Das Projekt wurde innerhalb des geplanten Zeitrahmens abgeschlossen. Es gab keine signifikanten Verzögerungen, und die einzelnen Phasen wurden wie geplant durchgeführt.


\subsection{Lessons Learned}
\label{\detokenize{sections/fazit:lessons-learned}}
\sphinxAtStartPar
Während des Projekts wurden mehrere Technologien und Konfigurationen implementiert, die über die Standardfunktionen von Django hinausgehen:
\begin{itemize}
\item {} 
\sphinxAtStartPar
\sphinxstylestrong{Redis Caching}: Implementierung eines effizienten Caching\sphinxhyphen{}Systems zur Optimierung der Ladegeschwindigkeiten.

\item {} 
\sphinxAtStartPar
\sphinxstylestrong{Django Signals}: Konfiguration und Nutzung von Signals zur automatischen Auslösung von Aktionen bei Modelländerungen.

\item {} 
\sphinxAtStartPar
\sphinxstylestrong{PostgreSQL}: Nutzung von PostgreSQL als Produktionsdatenbank für erweiterte Funktionen und bessere Skalierbarkeit.

\item {} 
\sphinxAtStartPar
\sphinxstylestrong{Django Simple History}: Implementierung zur Nachverfolgung von Änderungen an Modellen.

\item {} 
\sphinxAtStartPar
\sphinxstylestrong{WhiteNoise}: Konfiguration zur effizienten Bereitstellung von statischen Dateien.

\item {} 
\sphinxAtStartPar
\sphinxstylestrong{Automatische Migrationen}: Nutzung und Konfiguration von Django\sphinxhyphen{}Migrationen zur Verwaltung von Datenbankschemata.

\item {} 
\sphinxAtStartPar
\sphinxstylestrong{Mehrsprachigkeit}: Unterstützung für mehrsprachige Inhalte (DE/EN) in der Anwendung.

\item {} 
\sphinxAtStartPar
\sphinxstylestrong{Django\sphinxhyphen{}Admin}: Anpassung und Erweiterung des Django\sphinxhyphen{}Admin\sphinxhyphen{}Interfaces zur besseren Verwaltung der Inhalte.

\item {} 
\sphinxAtStartPar
\sphinxstylestrong{Sicherheitskonfigurationen}: Implementierung von Sicherheitsmaßnahmen wie CSRF\sphinxhyphen{}Schutz und sichere Cookie\sphinxhyphen{}Einstellungen.

\end{itemize}


\subsection{Ausblick}
\label{\detokenize{sections/fazit:ausblick}}
\sphinxAtStartPar
Projekt in Zukunft weiterentwickeln (z.B. geplante Erweiterungen)
\sphinxhyphen{} \sphinxstylestrong{Erweiterte Funktionen}: Implementierung zusätzlicher Funktionen wie erweiterte Benutzerrollen und Berechtigungen.
\sphinxhyphen{} \sphinxstylestrong{Frontend\sphinxhyphen{}Optimierungen}: Verbesserung der Benutzeroberfläche und der Benutzererfahrung.
\sphinxhyphen{} \sphinxstylestrong{Skalierbarkeit}: Weitere Optimierungen zur Unterstützung eines größeren Benutzeraufkommens und zusätzlicher Inhalte.
\sphinxhyphen{} \sphinxstylestrong{Sicherheitsmaßnahmen}: Implementierung zusätzlicher Sicherheitsfunktionen zum Schutz der Daten und der Anwendung.


\subsection{Referenzen}
\label{\detokenize{sections/fazit:referenzen}}\begin{itemize}
\item {} 
\sphinxAtStartPar
Siehe {\hyperref[\detokenize{sections/models:module-pages_app.models.Page}]{\sphinxcrossref{\sphinxcode{\sphinxupquote{pages\_app.models.Page}}}}} für die Implementierung des Page\sphinxhyphen{}Modells.

\item {} 
\sphinxAtStartPar
Weitere Details zur Cache\sphinxhyphen{}Strategie finden Sie in \sphinxcode{\sphinxupquote{pages\_app.cache}}.

\item {} 
\sphinxAtStartPar
Die Konfiguration von Django Signals ist in \sphinxcode{\sphinxupquote{pages\_app.signals.invalidate\_cache()}} dokumentiert.

\end{itemize}

\sphinxstepscope


\section{Projektantrag}
\label{\detokenize{sections/antrag:projektantrag}}\label{\detokenize{sections/antrag::doc}}
\sphinxAtStartPar
\sphinxstylestrong{Ausbildungsberuf}: Fachinformatiker/Fachinformatikerin (VO 2020)

\sphinxAtStartPar
\sphinxstylestrong{Fachrichtung}: Anwendungsentwicklung

\sphinxAtStartPar
\sphinxstylestrong{Prüfungsbezirk}: FIAN BSW 01 (AP T2V1)

\sphinxAtStartPar
\sphinxstylestrong{Teilnehmer}: Maximiliano Walter del Valle Santander
\sphinxstylestrong{Identnummer}: 1147414
\sphinxstylestrong{E\sphinxhyphen{}Mail}: \sphinxhref{mailto:max.santander@outlook.com}{max.santander@outlook.com}
\sphinxstylestrong{Telefon}: +49 174 759 0459

\sphinxAtStartPar
\sphinxstylestrong{Ausbildungsbetrieb}:
RIS Web\sphinxhyphen{} \& Software\sphinxhyphen{}Development GmbH \& Co. KG
Siemensstraße 9, 93055 Regensburg

\sphinxAtStartPar
\sphinxstylestrong{Projektbetreuer}: Brian Dymek
\sphinxstylestrong{E\sphinxhyphen{}Mail}: \sphinxhref{mailto:dymek@ris-development.de}{dymek@ris\sphinxhyphen{}development.de}
\sphinxstylestrong{Telefon}: +49 941 20001250


\subsection{Thema der Projektarbeit}
\label{\detokenize{sections/antrag:thema-der-projektarbeit}}
\sphinxAtStartPar
Entwicklung eines eigenen Backends für die neue Webseite von RIS Web\sphinxhyphen{} \& Software\sphinxhyphen{}Development GmbH \& Co. KG.


\subsection{1. Thema der Projektarbeit}
\label{\detokenize{sections/antrag:id1}}
\sphinxAtStartPar
Entwicklung eines eigenen Backends für die neue Webseite von RIS Web\sphinxhyphen{} \& Software\sphinxhyphen{}Development GmbH \& Co. KG.


\subsection{2. Geplanter Bearbeitungszeitraum}
\label{\detokenize{sections/antrag:geplanter-bearbeitungszeitraum}}\begin{itemize}
\item {} 
\sphinxAtStartPar
\sphinxstylestrong{Beginn}: 11.11.2024

\item {} 
\sphinxAtStartPar
\sphinxstylestrong{Ende}: 06.01.2025

\end{itemize}


\subsection{3. Ausgangssituation}
\label{\detokenize{sections/antrag:ausgangssituation}}
\sphinxAtStartPar
Die aktuelle Website von RIS basiert auf WordPress und erfüllt ihre grundlegende Funktion. Jedoch gibt es folgende Einschränkungen:
\begin{itemize}
\item {} 
\sphinxAtStartPar
WordPress erfordert die Nutzung allgemeiner Plugins, die oft nicht an die Anforderungen von RIS angepasst sind.

\item {} 
\sphinxAtStartPar
Viele Funktionen, die nicht benötigt werden, verursachen unnötigen Overhead beim Lernen und Bedienen.

\item {} 
\sphinxAtStartPar
Die Verwaltung von Inhalten in WordPress mit Templating hat sich als umständlich und unvorhersehbar erwiesen.

\end{itemize}


\subsection{4. Projektziel}
\label{\detokenize{sections/antrag:projektziel}}
\sphinxAtStartPar
Das Ziel des Projekts ist der Aufbau und die Implementierung eines Backends, das folgende Anforderungen erfüllt:
\begin{itemize}
\item {} 
\sphinxAtStartPar
\sphinxstylestrong{Volle Kontrolle}: Feinjustierung der Funktionen, unabhängig von allgemeinen Lösungen wie WordPress.

\item {} 
\sphinxAtStartPar
\sphinxstylestrong{Verschlankung}: Anpassung an die spezifischen Bedürfnisse von RIS.

\item {} 
\sphinxAtStartPar
\sphinxstylestrong{Effiziente Inhaltspflege}: Direkte und vereinfachte Verwaltung ohne Einschränkungen.

\item {} 
\sphinxAtStartPar
\sphinxstylestrong{Performance}: Server\sphinxhyphen{}seitiges Rendering und vollständiges Caching der Inhalte.

\end{itemize}


\subsection{5. Zeitplanung}
\label{\detokenize{sections/antrag:zeitplanung}}
\sphinxAtStartPar
\sphinxstylestrong{Projektplanung (8 Std)}:
\sphinxhyphen{} Ist\sphinxhyphen{}Analyse: Analyse der aktuellen Plugins, Templates, Ladezeiten und Performance.
\sphinxhyphen{} Soll\sphinxhyphen{}Konzept: Definition der Anforderungen und Funktionen.
\sphinxhyphen{} Lösungsansätze: Vergleich geeigneter Frameworks (Django, Laravel, Node.js).

\sphinxAtStartPar
\sphinxstylestrong{Entwurfsphase (16 Std)}:
\sphinxhyphen{} Datenbankmodell: Entwurf eines Datenbankschemas.
\sphinxhyphen{} Programmlogik: Definition der Backend\sphinxhyphen{}Interaktionen.

\sphinxAtStartPar
\sphinxstylestrong{Implementierung (36 Std)}:
\sphinxhyphen{} Benutzerverwaltung: Rollen, Authentifizierung und Berechtigungen.
\sphinxhyphen{} Serverseitiges Caching: Einführung eines robusten Caching\sphinxhyphen{}Systems.
\sphinxhyphen{} Backend\sphinxhyphen{}Logik: Implementierung der Datenverwaltung.

\sphinxAtStartPar
\sphinxstylestrong{Testphase (8 Std)}:
\sphinxhyphen{} Tests durchführen und Fehler beheben.

\sphinxAtStartPar
\sphinxstylestrong{Dokumentation (8 Std)}:
\sphinxhyphen{} Erstellung einer technischen Dokumentation.

\sphinxAtStartPar
\sphinxstylestrong{Abnahme (4 Std)}:
\sphinxhyphen{} Soll\sphinxhyphen{}Ist\sphinxhyphen{}Vergleich und abschließende Überprüfung.

\sphinxAtStartPar
\sphinxstylestrong{Gesamtstunden}: 80


\subsection{6. Anlagen}
\label{\detokenize{sections/antrag:anlagen}}
\sphinxAtStartPar
Keine.


\subsection{7. Präsentationsmittel}
\label{\detokenize{sections/antrag:prasentationsmittel}}
\sphinxAtStartPar
Laptop, Beamer, Presenter.


\subsection{8. Hinweis}
\label{\detokenize{sections/antrag:hinweis}}
\sphinxAtStartPar
Ich bestätige, dass der Projektantrag eigenständig angefertigt wurde und keine Betriebsgeheimnisse enthält. Personenbezogene Daten wurden nur mit Zustimmung der betroffenen Person verwendet.

\sphinxAtStartPar
Mit dem Absenden des Projektantrages bestätige ich weiterhin, dass der Antrag eigenständig von mir erstellt wurde und keine Plagiate enthält.


\renewcommand{\indexname}{Python-Modulindex}
\begin{sphinxtheindex}
\let\bigletter\sphinxstyleindexlettergroup
\bigletter{p}
\item\relax\sphinxstyleindexentry{pages\_app.models.base}\sphinxstyleindexpageref{sections/models:\detokenize{module-pages_app.models.base}}
\item\relax\sphinxstyleindexentry{pages\_app.models.Block}\sphinxstyleindexpageref{sections/models:\detokenize{module-pages_app.models.Block}}
\item\relax\sphinxstyleindexentry{pages\_app.models.MenuItem}\sphinxstyleindexpageref{sections/models:\detokenize{module-pages_app.models.MenuItem}}
\item\relax\sphinxstyleindexentry{pages\_app.models.Page}\sphinxstyleindexpageref{sections/models:\detokenize{module-pages_app.models.Page}}
\item\relax\sphinxstyleindexentry{pages\_app.models.PageBlock}\sphinxstyleindexpageref{sections/models:\detokenize{module-pages_app.models.PageBlock}}
\end{sphinxtheindex}

\renewcommand{\indexname}{Stichwortverzeichnis}
\printindex
\end{document}